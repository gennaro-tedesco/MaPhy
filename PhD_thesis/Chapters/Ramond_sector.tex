%*****************************************
\subsection{The Ramond sector}
\label{The Ramond sector}
%*****************************************
The real free Fermi field possesses another faithful 
representation of positive energy, the Ramond 
sector, as we have seen in \ref{repn of Fermi on the circle}, 
induced by the \ac{GNS} construction from the 
two-point function \eqref{Ramond}. Fields evaluated
in the Ramond sector will be expressed as 
$\ram{z}$. Obviously, as previously stated, their 
representation on the circle in terms of Fourier modes
has a cut at $z=-1$ and extends anti-periodically on the 
whole $\s$:
\[
\ram{z}=\sum_{n\in\Z}{\psi_{\textrm{R},}}_n \,z^{-n-1/2}.
\]
In principle one could just introduce a new field obtained 
by multiplying the actual Ramond field by $\sqrt{z}$, 
in order to ``cancel'' the cut: $\ram{z}\mapsto 
\sqrt{z}\cdot\ram{z}$ and so 
\[
\ram{z}=\sum_{n\in\Z}{\psi_{\textrm{R},}}_n \,z^{-n}
\]
with two-point function
\[
\omega_{\textrm{R}}(\ram{z}\ram{w})=
\frac{1}{2}\cdot\frac{z+w}{z-w}.
\]
Indeed, if we do so, the transformation law for the new 
defined conformal field under diffeomorphisms 
$z\mapsto \gamma(z)$ changes by an extra factor of $\sqrt{z/\gamma(z)}$.
Commutation relations between modes have the form 
$\ant{{\psi_{\textrm{R},}}_n,{\psi_{\textrm{R},}}_m}
=\delta_{n+m,0}$. In particular, the zero mode squares
to one: $2{\psi^2_{\textrm{R},}}_0=\bm{1}$.

\bigskip 
\begin{proposition}
A similar isomorphism like in \eqref{beta} can be 
introduced as 
\begin{equation}
\label{beta_ramond}
\beta_{\textrm{R}}\colon\pi_0(\alg{\I})\otimes^t 
\pi_{\textrm{R}}(\alg{\I})\to
\pi_{\textrm{R}}\left(\alg{\sqrt{I}}\right)
\end{equation}
taking the tensor product of fields in the vacuum and in 
the Ramond sector and defining
\begin{align}
\ram{z^2}\otimes^t\bm{1}_0&\mapsto\frac{1}{2}
\Big(\ram{z}+\ram{-z}\Big)\\[1ex]
\bm{1}_{\textrm{R}}\otimes^t\pi_0(\psi(z^2))&\mapsto\frac{1}{2z}
\Big(\ram{z}-\ram{-z}\Big).
\end{align}
This isomorphism still preserves the vacuum state in the 
form $\omega_{\textrm{R}}\circ\beta_{\textrm{R}}=
\omega_{\textrm{R}}\otimes\vac$.
\end{proposition}
\begin{proof}
We look again at the Fourier modes and the right hand 
sides present a relabelling ${\psi_{\textrm{R},}}_n\mapsto 
{\psi_{\textrm{R},}}_{2n}$ and ${\psi_{\textrm{R},}}_n\mapsto 
{\psi_{\textrm{R},}}_{2n+1}$, respectively;
this ensures that the correct commutation relations 
and two-point function directly follow.\qedhere
\end{proof}











%*****************************************
%*****************************************
%*****************************************
%*****************************************
%*****************************************