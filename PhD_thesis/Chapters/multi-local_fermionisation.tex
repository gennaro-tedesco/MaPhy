%*****************************************
\subsection{Multi-local fermionisation and 
gauge transformations}
\label{multi-local fermionisation}
%*****************************************
The multi-local isomorphism $\beta$ provides 
a correspondence between fermions localised 
essentially at our will. As we have seen, Fermi 
theories sort of automatically contain (non)-abelian
current algebras, due to the fact that these can 
be embedded using the standard quarks construction
from sufficiently many free fermions.

This feature, in its general behaviour, is referred to 
in the literature as ``fermionisation'', since it 
allows to express bosons (the currents) in terms 
of products of Fermi fields. In particular currents 
are expressed as Wick products (equation \eqref{current})
with subtraction of the vacuum expectation value;
since $\beta$ preserves the vacuum state it extends to
Wick products and therefore embeds 
the currents giving rise to a new feature which is
the delocalisation of the components of the current
itself. We shall be more precise showing the 
construction of such objects on the circle, because
most of the formulae drastically simplify.

\bigskip
Let us start from the vacuum representation and
take the case $N=2$; we look in particular at 
symmetric intervals, for the sake of simplicity, 
though the same construction can be easily generalised
with different coefficients eventually. In terms of the 
complex fermion the current is expressed as
\begin{equation}
\label{curr_complex}
j(z)\coloneqq \wick{\phi^*\phi}(z)=
i\,\wick{\psi_1\psi_2}(z);
\end{equation}
we are going to embed such formula with the help 
of \eqref{beta}. As so, we have (we evaluate on $z^2$
due to $\beta$) 
\begin{align*}
\beta(j(z^2))&=\beta\left(\wick{\phi^*\phi}(z^2)\right)\\
&=\beta\left(\phi^*(z^2)\phi(z^2)-\vac(\phi^*(z^2)\phi(z^2))\right)\\
&=\beta(\phi^*(z^2))\beta(\phi(z^2))-
  \vac\circ\beta(\phi^*(z^2)\phi(z^2))\\
&=\wick{\beta(\phi^*)\beta(\phi)}(z^2).
\end{align*}
The feature of $\beta$ to preserve the vacuum state means that 
it can be taken into the Wick product
due to commutativity $\beta\circ\wick{\blank}=\wick{\blank}
\circ\beta$. Consequently we have
\begin{align*}
\beta(j(z^2))&=\wick{\frac{1}{2z}
\left(\psi(z)-\psi(-z)\right)\frac{1}{2}
\left(\psi(z)+\psi(-z)\right)}\\
&=\frac{1}{4z}\,\wick{\psi(z)^2-
\psi(-z)\psi(z)+\psi(z)\psi(-z)-\psi(-z)^2}
\end{align*}
Fermions anti-commute and so $\psi(z)^2=\psi(-z)^2=0$;
moreover we have $-\psi(-z)\psi(z)=\psi(z)\psi(-z)$.
Making use of such relations we come to the 
\emph{multi-local fermionisation} formula on the 
circle:
\begin{equation}
\label{multi-local_fermionisation}
\beta(j(z^2))=\frac{1}{2z}\,\wick{\psi(z)\psi(-z)}
\end{equation}
The same expression, evaluated on the real line, looks like:
\[
\beta(j(q(x))= \frac{-i}{2x}\cdot\frac{(1-x^2)^2}{(1+x^2)}
\,\wick{\psi(x)\psi(-1/x)}\,.
\]
The current embedded with the isomorphism $\beta$
happens to be delocalised in two anti-podal points
on the circle, $z$ and $-z$. This feature justifies 
the term ``multi-local'' fermionisation (also in
\cite{Rehren:2012wa}) here and henceforth, because
the fermionisation is shared between pairs of 
different points. We notice that this new representation 
of the current is periodic on the circle 
under the change $z\to -z$;
in fact, both numerator (due to the anti-commutators)
and denominator acquire a minus sign, cancelling each
other altogether. Furthermore, since the expectation value 
of a Wick product is always zero, we still have 
$\vac\circ\beta (j(z^2))=0$. On the other hand the 
two-point function is 

\begin{align*}
\vac\circ\beta(j(z^2)j(w^2))&=
\frac{1}{4zw}\,\vac(\wick{\psi(z)\psi(-z)}\,
\wick{\psi(w)\psi(-w)})\\[1ex]
&=\frac{1}{4zw}\,\big(\vac(
\bcontraction{}{\psi(z)}{\psi(-z)\psi(w)}{\psi(-w)}
\bcontraction[2ex]{\psi(z)}{\psi(-z)}{}{\psi(w)}
\psi(z)\psi(-z)\,\psi(w)\psi(-w))\\
&\quad-\vac(
\bcontraction{}{\psi(z)}{\psi(-z)}{\psi(w)}
\bcontraction[2ex]{\psi(z)}{\psi(-z)}{\psi(w)}{\psi(-w)}
\psi(z)\psi(-z)\,\psi(w)\psi(-w))\Big)\\[1ex]
&=\frac{1}{4zw}\,\Big(-\vac(\psi(z)\psi(w))\cdot 
\vac(\psi(-z)\psi(-w))\\
&\quad+\vac(\psi(z)\psi(-w))\cdot
\vac(\psi(-z)\psi(w))\Big)\\[1ex]
&=\frac{1}{4zw}\cdot\frac{4zw}{(z^2-w^2)^2}
=\vac(j(z^2)j(w^2))
\end{align*}
that is, the two-point function is preserved too.

\bigskip 
The embedded current can be decomposed into Fourier modes
on the circle
\[
\beta(j(z^2))=\frac{1}{2z}\sum_{m,k\in\Z+1/2}^N
\wick{\psi_m\psi_k}(-1)^{-k-1/2}z^{-m-k-1}
\]
if $m+k=n$ is odd then $(-1)^{k-n-1/2}=
(-1)^{-k-1/2}$ and the sum vanishes because of 
the anti-commutation relations: each contribution
has its own opposite. Therefore the only allowed
powers of $m+k$ are even powers of the form $m+k=2p$, 
which lead us to 
\[
\beta(j_n)=\sum^{\infty}_{m=0}\psi_{n-m-1/2}\,
\psi_{n+m+1/2}(-1)^{n+m+1}.
\]
The complex fermion is invariant under gauge 
transformations generated by its own embedded 
currents: if $j(z)=\wick{\phi^*\phi}(z)$ then
Weyl-type operators are implemented by
smearing with suitable test functions $f\colon\s\to\R$
in order to obtain $W(f)=\e^{ij(f)}$. Gauge
transformations are then given by
\begin{align*}
\phi'(z)&=\alpha_f(\phi(z))=W(f)\phi(z)W(f)^*
=\e^{-if(z)}\phi(z)\\
{\phi^*}'(z)&=\alpha_f(\phi(z)^*)=W(f)\phi(z)^*W(f)^*
=\e^{if(z)}\phi(z).
\end{align*}
It is now very interesting to embed the gauge 
transformations themselves with the help of $\beta$,
in order to bring new delocalised gauge symmetries,
a brand new feature that we are going to present and 
fully exploit. If $\beta$ embedds the current then it
generates embedded gauge transformations on the free
fermion of the form
\begin{align*}
\beta(W(f))\psi(z)\beta(W(f))^*&=(\beta\circ\alpha_f\circ 
\beta^{-1})\psi(z)\\
&=(\beta\circ\alpha_f)(\phi(z^2)+z\,\phi(z^2)^*)\\
&=\beta(\alpha_f(\phi(z^2)) + z\,\alpha_f(\phi(z^2)^*))\\
&=\e^{-if(z^2)}\beta(\phi(z^2))+z\,\e^{if(z^2)}
\beta(\phi(z^2)^*)
\end{align*}
eventually we end up with
\begin{equation}
\label{emb_gauge}
\beta(W(f))\psi(z)\beta(W(f))^*=
\cos{f(z^2)}\psi(z)-i\sin{f(z^2)}\psi(-z).
\end{equation}
The new remarkable feature is the bilocal mixing
of $\psi(z)$ and $\psi(-z)$, reflecting the 
non-locality of the isomorphism $\beta$.
Of course the same calculations can be performed 
in the situation $N>2$: in this case non-abelian 
currents of the form $j_{rs}(z)\coloneqq
\wick{\phi_r^*\phi_s}(z)$ can be embedded and we
obtain representations of all these theories in 
the Fock space of a single real free fermion. In the 
many interval case $\beta$ delocalises the fields
onto $2N$ points (equations \eqref{beta_any} and,
in general, \eqref{map_beta_general}) and therefore 
expressions like $\wick{\phi_r^*\phi_s}(z)$ present 
sums of fields in pairwise coupled points
$\psi(z_j)\psi(z_l)$ with position dependent 
coefficients. Gauge transformations change 
accordingly, having multi-local contributions 
from different points. 

\bigskip 
The same argument can be run in the Ramond sector:
the fact that the current $j(z)\coloneqq 
\wick{\phi^*\phi}(z)=i\,\wick{\psi_1\psi_2}(z)$
satisfies commutation relations in purely algebraic
and independent of the representation. As a consequence 
one can then take the two fields $\psi_1$ and $\psi_2$ in two
different representations and \emph{twist} the 
product. In fact, by taking $\psi_1$ in the vacuum
sector and $\psi_2$ in the Ramond one, the 
isomorphism \eqref{beta_ramond} embeds the 
resulting current into the Ramond algebra 
$\pi_{\textrm{R}}\Big(\alg{\sqrt{\I}}\Big)$. The Wick 
product here is defined as the subtraction with 
respect to the corresponding Ramond two-point function; 
the result is, in the compact picture:
\[
\beta_{\textrm{R}}(j(z^2))=\frac{1}{2i z^2} \cdot 
\wick{\ram{z}\ram{-z}}_{\textrm{R}}.
\]
The new current, as expected, is delocalised at two 
anti-podal points in the Ramond representation; the 
Wick product is essentially the standard operator 
product, because the Ramond expectation value vanishes
at the case at hand in the points $z,-z$. Also, the 
formula changes sign under $z\to -z$, i. e. it is
anti-periodic in the variable $z^2$, expressing the fact
that the new current is twisted. Of course, the one-point 
function is still zero, whereas the two-point function 
becomes now:
\begin{align*}
\omega_{\textrm{R}}\circ\beta_{\textrm{R}}
(j(z^2)j(w^2))&= -\frac{1}{4z^2w^2}\,\omega_{\textrm{R}}
(\wick{\ram{z}\ram{-z}}\\
&\wick{\ram{w}\ram{-w}})\\[1ex]
&=\frac{1}{2zw}\cdot\frac{z^2+w^2}{(z^2-w^2)^2}
\end{align*}
after making use of the usual Wick contractions within the 
expectation value. This formula has been previously
mentioned by \cite{Ang:2011} as the ``twisted'' 
representation of the current, in the context of
vertex operator algebras (we refer the reader to 
the references therein).










%*****************************************
%*****************************************
%*****************************************
%*****************************************
%*****************************************