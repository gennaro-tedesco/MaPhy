%************************************************
\chapter{Currents models and embeddings}\label{ch:Currents}
\minitoc\mtcskip
%************************************************

%*******************************************************
 \section{Loop groups}
 \label{Loop groups}
 We shall now briefly describe the construction of nets
 of von Neumann algebras on the circle in the framework
 of loop groups as shown in \cite{PS1986}. 
 
 Let $G$ be a compact Lie group whose Lie algebra is denoted
 by $\mathfrak{g}$: the set of all smooth maps
 $LG\coloneqq\set{g\mid \s\to G}$ equipped with
 pointwise multiplication $(g\cdot h)(z)=g(z)h(z)$
 is an infinite dimensional Lie group called the loop
 group. As such, it possesses a Lie algebra which can
 be shown, as expected, to be the set of all maps 
 $L\mathfrak{g}=\set{g\mid \s\to \mathfrak{g}}$. 
 As set of maps, both of them can be equipped with the standard
 topologies of uniform convergence and differential structure,
 and thus a smooth map $\textrm{exp}\colon L\mathfrak{g}
 \to LG$ exists and is a local homeomorphism in the connected
 neighbourhood near the identity.
 
 A localised subgroup $L_{\I}G$ is the set of all
 such functions taking the trivial value $\bm{1}_G$ 
 outside the interval $\I$, essentially  
 \[
 LG\coloneqq\set{g\mid \s\to G}\qquad
 L_{\I}G\coloneqq \set{g\mid g(z)
 =\bm{1}_G\in G, \text{ if }z\notin\I}.
 \]
 Since now on we shall focus on projective unitary 
 representations of the compact Lie group $G$, where 
 ``projective'' means that products are preserved up to a 
 complex phase. In order to make formal sense 
 of such a concept we introduce the following
 definition:
 \begin{definition}[2-cocycle]
 Let $G$ be a group. A 2-cocycle is a map $\omega\colon
 G\times G\to\s$ satisfying, however we choose $f,g,h\,\in G$
 \[
 \omega(f,g)\omega(fg,h)=\omega(f,gh)\omega(g,h);
 \]
 also, $\omega$ must be trivial on the identity element
 $\omega(\bm{1}_G,g)=\omega(g,\bm{1}_G)=1$. By pointwise
 multiplication the set of all 2-cocycles forms a group. 
 Moreover, if there exists
 $\beta\colon G\to \s$ such that
 \[
 \omega(f,g)=\frac{\beta(f)\beta(g)}{\beta(fg)}
 \]
 then the 2-cocycle is said to be a coboundary. 
 \end{definition}
 With the help of the above definition we can therefore
 define projective representations of a loop group as
 maps from the groups itself into the set of unbounded
 operators on some Hilbert space preserving products
 \[
 W(g_1)W(g_2)=\omega(g_1,g_2)
 W(g_1 g_2),\quad g_1,g_2\,\in LG
 \]
 $\omega$ being a cocycle of $LG$.
 The assignment $\I\to\alg{\I}$ defined as
 \[
 \alg{\I}\coloneqq\set{W(g)\mid g\in L_{\I}G}''
 \]
 defines a net of local algebras whenever the cocycle
 $\omega$ is local. Nets of von Neumann algebras defined
 out of loop groups representations have the property
 to only have finitely many inequivalent representations.
 
 \bigskip
 In the field theoretical setting such construction
 are realised by taking the analogue of the Weyl operators
 for current algebras. Let $\tau_a$ be a basis 
 of the Lie algebra $\mathfrak{g}$ of G and define 
 $f(z)=\tau_a f^a(z)$. We define the smeared current
 $j(f)\coloneqq\oint_{\s}\dd z\, j_a(z)f^a(z)$
 \[
 \comm*{j(f_1),j(f_2)}=j\comm*{f_1,f_2} + 
 k\oint_{\s}\dd z\,\omega(f_1,f_2)(z),
 \quad f_1, f_2\in\,L\mathfrak{g}
 \]
 (notice here that, despite the same notation,
 $\omega$ is an additive cocycle playing the infinitesimal
 version of the previous one introduced for Weyl relations).
 The corresponding Weyl operators are
 \[
 W(g)\coloneqq\e^{ij(f)}\qquad g(z)=\exp (f)(z)
 \]
 whose collection generates the local net of von Neumann
 algebras~$\alg{\I}$ as $\textrm{supp}\ f\subset \I$.
 
 In this context gauge transformations $\gamma$ are defined 
 as automorphisms $\gamma\colon\alg{\I}\to\alg{\I}$ 
 that preserve every local subalgebra and they may 
 be inner implemented by means of the unitaries $W(g)$.
 For instance, on currents, $\gamma(J^a)=W(g)J^a W(g)^*$ 
 acts as
 \[
 \gamma(J^a\tau_a)=g(z)^{-1} J^a\tau_a g(z) +
 k\,g(z)^{-1}\partial_z g(z);
 \]
 the matrices $\tau_a$ form a basis in the Lie algebra 
 $\mathfrak{g}$ and the transformation law
 for the actual fields (the currents) follow by multiplying
 and taking traces with respect to $\tau_a$. For example, in
 case $G=\textrm{SU}(2)$ a basis for the Lie algebra is 
 given in terms of the Pauli matrices and thus, after using
 \[
 \sigma_i\,\sigma_j=i\,\epsilon_{ij}^{\phantom{ij}k}\sigma_k
 +\delta_{ij}\bm{1}
 \]
 we obtain, taking into account that Pauli matrices
 are traceless themselves
 \[
 \gamma(J^a)=f_{ab}\,J^b +\frac{1}{2}\,
 i\tr\left(g(z)^{-1}\partial_z g(z)\sigma_a\right)
 \] 
 with $f_{a}^{\phantom{a}b}\,\sigma_b=g(z)^{-1}\sigma^a g(z)$.
 On the other hand, let now $h\colon\s\to G$ be a function
 periodic up to a central element $h_0\in Z(G)$. 
 Any transformation of the form
 \[
 \gamma(J)=\ad h^{-1} (J) + h^{-1} \dd h
 \]
 still fulfils the requirements to be an
 automorphism, because the central element
 cancels out; nevertheless it is not implemented
 by unitaries of the form $W(h)$ because 
 $h$ is no more an element of the loop group.
 Therefore, by using elements in $Z(G)$
 we can construct automorphisms which are no
 more implemented by unitaries but which are still 
 inner symmetries. This is an interesting feature in the
 context of representation theory of loop groups,
 because the composition of states with such automorphisms
 gives rise to inequivalent representations and all
 simple sectors are of this form. As a 
 consequence, different sectors arise according to
 how many central elements the Lie group $G$ has.
 Thus compact Lie groups with trivial centre only have
 one simple sector (the vacuum sector).
 \begin{example}[non-trivial sectors]
 Let $\I_1$ and $\I_2$ be two intervals on the circle
 such that their intersection is the union of two
 disjoint intervals $\textrm{J}_1$ and $\textrm{J}_1$:
 \begin{figure}[htbp]
 \centering
 \def\first{1.05}
 \def\second{1.15}
 \def\third{1.25}
 \def\rad{0.95}
 \begin{tikzpicture}[scale=1.5]
 % circle
 \path (0,0) coordinate (O);
 \draw [style=dashed](O) circle (\rad);
 % arcs
 \draw [line width=0.4mm, color=navy!60!, domain=-20:200] 
       plot ({\first*cos(\x)}, {\first*sin(\x)});
 \draw [line width=0.4mm, color=navy!40!, domain=160:380] 
       plot ({\second*cos(\x)}, {\second*sin(\x)});
 \draw [line width=0.4mm, color=navy!80!, domain=160:200] 
       plot ({\third*cos(\x)}, {\third*sin(\x)});
 \draw [line width=0.4mm, color=navy!80!, domain=-20:20] 
       plot ({\third*cos(\x)}, {\third*sin(\x)});
 \node at (90:\third) {$\I_1$};
 \node at (270:1.4) {$\I_2$};
 \node at (180:1.6) {$\textrm{J}_1$};
 \node at (0:1.55) {$\textrm{J}_2$};
 %lines for the intervals
 \draw [dotted] (-20:0.4) --(-20:1.5);
 \draw [dotted] (20:0.4) --(20:1.5);
 \draw [dotted] (160:0.4) --(160:1.5);
 \draw [dotted] (200:0.4) --(200:1.5);
 \end{tikzpicture}
 \end{figure}
 moreover, let $\alg{\textrm{J}_1}$ and $\alg{\textrm{J}_2}$ come 
 accompanied with two different localised representations
 $\pi_1\left(\alg{\textrm{J}_1}\right)$, 
 $\pi_2\left(\alg{\textrm{J}_2}\right)$ different
 from the defining vacuum representation. If now
 $U\colon\pi_1\to\pi_2$ is a map intertwining such
 representations, namely $U\,\pi_1(a_1)=\pi_2(a_1)\,U$,
 then $U$ belongs, by Haag duality, to both
 $\pi_0\left(\alg{\I_1}\right)$ and $\pi_0\left(\alg{\I_2}\right)$;
 yet, the operator $U$ may differ when evaluated 
 in $\pi_1$ and $\pi_2$, that is
 $\pi_1(U)\neq\pi_2(U)$. One can prove, using 
 Doplicher-Haag-Roberts theory of localised endomorphisms,
 that in case the net $\I\to\alg{\I}$ has only the vacuum
 sector no such problem occurs and the vacuum representation
 is always faithful. This also helps to globally define the
 whole algebra $\alg{\s}$ as the $C^*$-algebra generated by all
 the $\vee_{\I}{\pi_0}(\alg{\I})=\bh$.
 \end{example}
 We want to remark once more that the quarks construction, as
 showed in the previouos chapters, provides, in the field 
 theoretical setting, the relation 
 between Fermi fields and currents expressed 
 as Wick products thereof.
 
 
 \section{Currents models}
 \label{Currents models}
 We have previously seen that the isomorphism $\beta$ provides
 a map $\beta\colon\mathcal{A}^{(N)}(\I)\to
 \mathcal{A}(\sqrt[N]{\I})$ preserving the vacuum state
 and its representation $\pi_0\circ\beta=\pi_0$. In
 the particular case of $N=2$ a complex Fermi field
 localised in one interval $\I$ is ``decomposed'' into
 its symmetric and antisymmetric part
 \begin{align*}
 \phi(z^2)&=\frac{1}{\sqrt{2}}\left(\psi^{(1)}(z^2)
           +i\psi^{(2)}(z^2)\right)\mapsto\frac{1}{2}
           \left(\psi(z)+\psi(-z)\right)\\    
 \phi^*(z^2)&=\frac{1}{\sqrt{2}}\left(\psi^{(1)}(z^2)
           -i\psi^{(2)}(z^2)\right)\mapsto\frac{1}{2z}
           \left(\psi(z)-\psi(-z)\right).
 \end{align*}
 The resulting representation is a twisted 
 representation $\pi_0\otimes^t\pi_0$ 
 because $\beta$ intrinsically carries
 a twist on some fields, namely $\beta\circ\rot(2\pi)
 =\beta\circ\wp$ where $\wp$ is the flip automorphism
 flipping the tensor product $\wp\colon\mathcal{A}
 \otimes\mathcal{A}\to\mathcal{A}\otimes\mathcal{A}$ 
 such that $\wp(x\otimes y)=y\otimes x$.
 
 The idea is now to extend this map to embedded models,
 as well as currents and stress-energy tensor, trying to 
 preserve its features. We are then looking for an
 isomorphism which gives a correspondence
 $\alg{\I}\otimes\alg{\I}\to\alg{\sqrt{\I}}$
 also at the level of currents and stress-energy tensor,
 decomposing the fields into their symmetric and
 antisymmetric parts. Since the restriction of
 $\beta$ does not, in general, preserve this embedded
 subalgebras we should expect an additional gauge
 transformation to compose $\beta$ with in order to
 achieve the result.
 
 \bigskip 
 Let us denote by $\mathcal{A}^J(\I)$ the algebra of
 currents localised in the interval $\I$. The purpose
 is to explicitly construct a map $\iota\colon
 \mathcal{A}^J(\I)\otimes\mathcal{A}^J(\I)\to
 \mathcal{A}^J(\sqrt{\I})$ making use of $\beta$.
 We start by taking two real Fermi fields $\psi(z)$ and
 $\psi'(z)$ localised in $\sqrt{\I}$ 
 and apply $\beta^{-1}$ in order to obtain
 two complex Fermi fields $\phi(z^2), \phi'(z^2)$ 
 localised $\I$. These fields can in turn be decomposed
 into their respective real and imaginary parts as
 \begin{align*}
 \phi(z^2)&=\frac{1}{\sqrt{2}}\left(\psi^{(1)}(z^2)
           +i\psi^{(2)}(z^2)\right)\\    
 \phi'(z^2)&=\frac{1}{\sqrt{2}}\left(\psi^{(3)}(z^2)
           +i\psi^{(4)}(z^2)\right)
 \end{align*}
 and thus we have generated four real Fermi fields,
 $\psi^{(1)},\ldots,\psi^{(4)}$. Combinations of such
 fields can be used to generate current algebras models
 with gauge group $\textrm{O}(4)$, since in 
 principle we can combine any two Fermi fields into
 Wick products $\wick{\psi^i\psi^j}$; in particular
 the construction runs as follows:
 let us take the $\textrm{U}(1)=\textrm{SU}(2)$ 
 current constructed out
 of the combination of the two initial fields $j(z)\coloneqq
 2i\,\wick{\psi \psi'}(z)$ and embed by $\beta^{-1}$, taking
 into account the inverse formula in 
 \cite{Rehren:2012wa}.
 We have
 \begin{align*}
 \beta^{-1}(\psi(z)\psi'(z))&=\left(\phi(z^2)+z\phi^*(z^2)\right)
 \left(\phi'(z^2)+z\phi'^*(z^2)\right)\\
 \beta^{-1}(\psi(z)\psi'(z)-\psi(-z)\psi'(-z))
 &=2z\,\phi^*(z^2)\phi'(z^2)+2z\,\phi(z^2)\phi^*(z^2);
 \end{align*}
 as we see, only terms coupling hermitian products of $\phi^*\phi$
 appear, and thus we may conclude this current is neutral,
 the total charge being zero. Everything can be expressed
 in terms of the initial four fermions, and since $\beta$ 
 preserves the vacuum state and
 then the Wick products, we find for the even modes of such
 current
 \[
 \beta^{-1}\left(z\,j(z) -z\,j(-z)\right)=2 z^2 \left(J_{13}(z^2)
 +J_{24}(z^2)\right)
 \]
 with $J_{13}(z)\coloneqq 2i\,\wick{\psi^{(1)}\psi^{(3)}}(z)$
 and similarly for $J_{24}(z)$. The odd modes, instead,
 present charged combinations $\phi(z^2)\phi'(z^2) +
 z^2\,\phi^*(z^2)\phi'^*(z^2)$, giving rise to
 \[
 \beta^{-1}\left(z\,j(z) +z\,j(-z)\right)=J_+(z^2) +
 z^2\,J_-(z^2) 
 \]
 with $J(z)\coloneqq J_{13}(z)-J_{24}(z) +i
 \left(J_{14}(z)+J_{23}(z)\right)$. In a more compact
 way the above relations can be written as
 \begin{align*}
 \beta^{-1}(\text{even})&=2 z^2\,J_0(z^2)\\
 \beta^{-1}(\text{odd})&=J(z^2) + z^2\,J^*(z^2).
 \end{align*}
 Commutation relations between these currents show
 particular features: the commutator $\comm{J_+,J_-}$
 produces the third generator of $\mathfrak{su(2)}$ current algebra with
 $J_3\coloneqq J_{12}+J_{34}$, yet both $J_+,J_-$ commute with
 $J_0$. The structure is the one of a $\mathfrak{u(2)}\subset
 \mathfrak{o(4)}$ current
 algebra $\mathfrak{u(2)}=\mathfrak{su(2)}\oplus 
 \mathfrak{u(1)}$ where $J_0$ plays the
 role of the diagonal part in $\mathfrak{u(1)}$, the rest being
 the $\mathfrak{su(2)}$ current algebra. However, the action of
 $\beta^{-1}$ only gives back the commuting
 currents $J_0$ and $J_{\pm}$. We may as well reverse the picture
 and look at how the commuting currents, $J_0$ and $J_{\pm}$,
 localised in $\mathcal{A}^J(\I)$
 are decomposed into symmetric and antisymmetric 
 part of a single current localised in 
 $\mathcal{A}^J(\sqrt{\I})$.
 
 \bigskip
 With the help of suitable gauge transformations we can reduce
 the combination $J_+(z)+z\,J_-(z)$ to just a single current.
 $\textrm{SU}(2)$ gauge transformations $\gamma$ on
 Fermi fields transform the embedded currents
 $J^a(x)=\wick{\psi^* \sigma^a \psi}(x)$ as
 \[
 \gamma(J^a)\sigma_a=g(z)^{-1} J^a\sigma_a g(z) +
 k\,g(z)^{-1}\partial_z g(z)
 \]
 where $g\in\textrm{SU}(2)$ and 
 transformations for the actual fields $J^a$ follow
 by multiplying both sides by $\sigma_c$ and
 taking traces. Also, use $\sigma_a\sigma_b =
 i\varepsilon_{ab}^{\phantom{ab}c}\,\sigma_c +
 \delta_{ab}\bm{1}$. We obtain eventually
 \begin{equation}
 \label{GTcurrents}
 \gamma(J^a)=f^a_{\phantom{a}b}\,J^b +\frac{1}{2}\,
 \kappa\,\tr\left(g(z)^{-1}\partial_z g(z)\sigma^a\right)
 \end{equation}
 with $f^a_{\phantom{a}b}(z)\,\sigma^b=g(z)^{-1}\sigma^a g(z)$.
 Such gauge transformations are automorphisms of the
 algebras, though they may not preserve the vacuum state.
 Exploiting the above equation with the group element
 $g(z)=\e^{-i(\varphi/4)\sigma_3}
 \e^{-i(\varphi/4)\sigma_2}$
 gives back exactly
 \[
 \gamma\left(J_3(z^2)\right)=\sqrt{z}\,
 \left(J_+(z^2)+z^2\,J_-(z^2)\right).
 \]
 Moreover, since $J_0$ plays the role of the diagonal
 part $\mathfrak{u(1)}$ in $\mathfrak{u(2)}=\mathfrak{su(2)}
 \oplus \mathfrak{u(1)}$ the map $\gamma$
 can be regarded as acting diagonally on $J_0$. Therefore
 we have
 \begin{align}
 \label{J0J3}
 \left(\beta\circ\gamma\right)\left(J_0(z^2)\right)&=\frac{1}{2z}\,
 \left(j(z)-j(-z)\right)\\
 \left(\beta\circ\gamma\right)\left(J_3(z^2)\right)&=\frac{1}{2z}\,
 \left(j(z)+j(-z)\right).
 \end{align}
 Defining $\iota=\beta\circ\gamma$ gives us the map we looked for
 at the level of the currents. Interestingly enough, this map
 produces an anti-periodic field when acting on $J_3$. Therefore
 \[
 \iota\colon\mathcal{A}^J(\I)\otimes\mathcal{A}^J(\I)\to
 \mathcal{A}^J(\sqrt{\I}).
 \]
 The scenario we are dealing with looks now like, according 
 to the scaling dimension:
 \begin{align*}
 {\text{Fermi}}^{\ d=\frac{1}{2}}& \qquad
             \begin{matrix}
             \phi(z^2)\\[1ex]
             \phi^*(z^2)
             \end{matrix}\
             \overset{\beta}{\longrightarrow}\
             \begin{matrix}
             \psi(z)+\psi(-z)\\[1ex]
             z^{-1}\left(\psi(z)-\psi(-z)\right)
             \end{matrix}\\[3ex]
 {\text{Currents}}^{\ d=1}&\qquad
             \begin{matrix}
             J_0(z^2)\\[1ex]
             J_3(z^2)
             \end{matrix}\
             \xrightarrow[]{\beta\,\circ\,\gamma}
             \begin{matrix}\
             z^{-1}\left(j(z)-j(-z)\right)\\[1ex]
             z^{-1}\left(j(z)+j(-z)\right)
             \end{matrix}\\
 \end{align*}
 Of course, the twist emerges once we compose $\beta$ with
 $\rot(2\pi)$ and this in turn emerges from the mere
 commutation relations of any chiral field with the
 rotations $\e^{2\pi iL_0}$,
 \[
 i\comm{L_0,\phi(z)}=i(z\partial_z +h)\phi(z)
 \]
 which integrates to
 \[
 \e^{it L_0}\phi(z)\e^{-itL_0}=\e^{ith}\phi(\e^{it}z).
 \]
 This means that for scaling dimension $h=1/2$ we have a minus 
 sign at $t=2\pi$ if we evaluate fields in the vacuum 
 representation, while no minus sign occurs in the Ramond
 representation, since the latter presents an additional
 $\sqrt{z}$ which absorbs the $-1$. In contrast, no minus
 sign may appear for local fields with integer scaling dimension.  
 
 \bigskip
 Now, if we start from non-abelian current algebra
 with level $\kappa$ and structure constants 
 $f_{ab}^{\phantom{ab}c}$, we can easily construct 
 models with twice the level just by taking
 the symmetric and anti-symmetric parts. In detail we 
 start with \cite{Fuchs:1992}
 \begin{equation}
 \label{cur_alg_z}
 \comm*{J^a(z),J^b(w)}=f^{ab}_{\phantom{ab}c}\,J^c(z)\frac{1}{w}
 \sum_{n\in\Z}\left(\frac{z}{w}\right)^n
 +\frac{1}{zw}\,\kappa\,h^{ab}\sum_{n\in\Z}
 \left(\frac{z}{w}\right)^n
 \end{equation}
 and construct
 \begin{align*}
 J^{2\kappa}_a(z^2)&\coloneqq J^{c}_a(z^2)\otimes\bm{1} +
 \bm{1}\otimes J^{c}_a(z^2)\\
 \Delta^{2\kappa}_a(z^2)&\coloneqq J^{c}_a(z^2)\otimes\bm{1} -
 \bm{1}\otimes J^{c}_a(z^2)
 \end{align*}
 clearly both fields belong to $\mathcal{A}^J(\I)\otimes
 \mathcal{A}^J(\I)$. These quantities satisfy current
 algebras commutation relations like \eqref{cur_alg_z}
 with twice the central charge
 \begin{align*}
 \comm*{J_a^{2c}(z^2),J_b^{2c}(w^2)}&=f_{ab}^{\phantom{ab}c}
 \,J_c^{2c}(z^2)
 \frac{1}{w^2}\sum_{n\in\Z}\left(\frac{z}{w}\right)^{2n}\\
 &+\frac{1}{z^2 w^2}\,2c\,h_{ab}\sum_{n\in\Z}n
 \left(\frac{z}{w}\right)^{2n}
 \end{align*}
 as for the other combinations
 \begin{align*}
 \comm*{J_a^{2c}(z^2),\Delta_b^{2c}(w^2)}&=f_{ab}^{\phantom{ab}c}
 \Delta_c^{2c}(z^2)\frac{1}{w^2}\sum_{n\in\Z}
 \left(\frac{z}{w}\right)^{2n+1}\\[2ex]
 \comm*{\Delta_a^{2c}(z^2),\Delta_b^{2c}(w^2)}&=f_{ab}^{\phantom{ab}c}
 \,J_c^{2c}(z^2)
 \frac{1}{w^2}\sum_{n\in\Z}\left(\frac{z}{w}\right)^{2n+1}\\
 &+\frac{1}{z^2 w^2}\,2c\,h_{ab}\sum_{n\in\Z}(n+1/2)
 \left(\frac{z}{w}\right)^{2n+1}
 \end{align*}
 These commutation relations happen to be satisfied by the 
 odd and even modes of a single current localised in $\sqrt{\I}$,
 namely the assignment
 \begin{align}
 \alpha\left(J_a^{2c}(z^2)\right)&=\frac{1}{2z}
 \left(j_a(z)-j_a(-z)\right)\label{J2c}\\
 \alpha\left(\Delta_a^{2c}(z^2)\right)&=\frac{1}{2z}
 \left(j_a(z)+j_a(-z)\right)\label{delta2c}
 \end{align}
 preserves the commutation relations, as easily seen
 by summing up the Fourier modes. In detail let us 
 restrict to the abelian case: the Fourier decomposition
 of the current $J(z)=\sum_{n\in\Z}j_n z^{-n-1}$ gives
 \begin{align*}
 \alpha\left(J^{2c}_n\right)&=j_{2n}^c,\quad n\in\Z\\
 \alpha\left(\Delta^{2c}_{\nu}\right)&=j_{2n}^c\quad 
 \nu\in\Z+1/2
 \end{align*}
 which, by using $\comm{j_n,j_m}=n\,\delta_{n+m,0}\,c\,\bm{1}$ are
 seen to satisfy the commutation relations
 \begin{align*}
 \comm*{\alpha\left(J^{2c}_n\right),\alpha\left(J^{2c}_m\right)}
 &= n\,\delta_{n+m,0}\,2c\,\bm{1} = 2\,\alpha\comm*{j_n,j_m}\\
 \comm*{\alpha\left(J^{2c}_n\right),
 \alpha\left(\Delta^{2c}_{\nu}\right)}
 &= n\,\delta_{n+\nu,0}\,2c\,\bm{1} = 2\,\alpha\comm*{j_n,j_{\nu}}\\
 \comm*{\alpha\left(\Delta^{2c}_{\mu}\right),
 \alpha\left(\Delta^{2c}_{\nu}\right)}
 &= \mu\,\delta_{\mu+\nu,0}\,2c\,\bm{1} = 
 2\,\alpha\comm*{j_{\mu},j_{\nu}}.
 \end{align*}
 Summing up the Fourier series we obtain
 \begin{align*}
 \comm*{\alpha\left(J^{2c}(z^2)\right),\left(J^{2c}(w^2)\right)}&
 =\alpha\comm*{\left(J^{2c}(z^2)\right),\left(J^{2c}(w^2)\right)}\\
 \comm*{\alpha\left(J^{2c}(z^2)\right),\left(\Delta^{2c}(w^2)\right)}&
 =\alpha\comm*{\left(J^{2c}(z^2)\right),\left(\Delta^{2c}(w^2)\right)}\\
 \comm*{\alpha\left(\Delta^{2c}(z^2)\right),\left(\Delta^{2c}(w^2)\right)}&
 =\alpha\comm*{\left(\Delta^{2c}(z^2)\right),\left(\Delta^{2c}(w^2)\right)}.
 \end{align*} 
 In the abelian case
 the right hand side coincides with the quantities \eqref{J0J3} 
 we previously calculated as $\beta\circ\gamma$, therefore we
 may write 
 \begin{align*}
 \alpha\left(J^{2c}(z^2)\right)&=\left(\beta\circ\gamma\right) 
                                 \left(J_0^{2c}(z^2)\right)\\
 \alpha\left(\Delta^{2c}(z^2)\right)&=\left(\beta\circ\gamma\right) 
                                 \left(J_3^{2c}(z^2)\right)
 \end{align*}
 meaning
 \begin{align}
 \left(\alpha^{-1}\circ\beta\circ\gamma\right)\left(J_0(z^2)\right)&=
 J(z^2)\otimes\bm{1}+\bm{1}\otimes J(z^2)\\
 \left(\alpha^{-1}\circ\beta\circ\gamma\right)\left(J_3(z^2)\right)&=
 J(z^2)\otimes\bm{1}-\bm{1}\otimes J(z^2)
 \end{align}
 with the currents on the left hand side having twice the central
 charge of the currents on the right hand side.
 
 \subsection{The Kac-Frenkel construction}
 \label{Kac-Frenkel}
 We have seen in the previous section that starting
 from two real Fermi fields the inverse action of
 $\beta^{-1}$ gives back two complex Fermi fields which
 in turn can be decomposed into their real and imaginary
 parts. This brings us four Fermi fields whose 
 combinations construct a non-abelian current model. 
 In particular we can get a $\mathfrak{u}(2)$ current
 model constituted by the currents
 \begin{align*}
 J_0&=J_{13} + J_{24}\\
 J&=J_{13}-J_{24}+i\left(J_{14}+J_{23}\right)\\
 J_1&=J_{13}-J_{24}\\
 J_2&=J_{14}+J_{23}\\
 J_3&=J_{12}+J_{34}
 \end{align*}
 whose commutation relations are
 \begin{align}
 \comm{J_i,J_j}&=\epsilon_{ij}^{\phantom{ij}k}J_k
 \quad i,j,k=1,2,3\\
 \comm{J_i,J_0}&=0
 \end{align}
 giving raise to $\mathfrak{u}(2)=\mathfrak{su}(2)$ (the former)
 $\oplus \mathfrak{u}(1)$ (the latter). This also gives back
 the commutations relations $\comm{J,J^*}$ with $J=J_1
 +i\,J_2$.
 
 \bigskip
 The same $\mathfrak{u}(2)$ algebra can be derived
 by using the Kac-Frenkel construction (\cite{Kac:VA})
 out of two commuting currents $J_0$ (playing the role
 of the diagonal $\mathfrak{u}(1)$ current) and $J_3$ as 
 follows: the unitary Weyl operators on currents
 $W(f)=\e^{ij(f)}$ evaluated on sharp test functions
 $G_u(x)=q\cdot\theta(x-u)$ become the ``vertex operators''
 (\cite{LRboundary:2009})
 \[
 V_{q}(x)=
 \wick{\e^{iq \displaystyle{\int_{\infty}^x \dd u\,j(u)}}}\,;
 \]
 the operators $V_{\pm \sqrt{2}}(x)$ can be decomposed as  
 \[
 J^{\pm}(x)\coloneqq
 \wick{\e^{\pm\sqrt{2} i \displaystyle{\int_{\infty}^x \dd u\,j(u)}}}=
 J^1(x)\pm iJ^2(x)
 \]
 and $J_1,J_2$, together with $J_3= j/\sqrt{2}$, generate
 an $\mathfrak{su}(2)$ current algebra and $J_0$ plays
 the role of the $\mathfrak{u}(1)$ contribution. After
 performing such construction care must be taken to the
 fact that the vertex operators in general do not act
 on the same Hilbert space as the constituting currents.
 
 \section{Stress-energy tensor models}
 The same game can be played with the stress-energy tensor 
 and its Virasoro generators. Starting from a single
 Fermi field $\psi(z)$ one can construct the related stress-energy
 tensor (again following the quarks construction) as
 \[
 T^{c=1/2}(z)=\frac{-1}{4\pi}\,\wick{\psi \partial_z \psi}(z)
 =\frac{-1}{8\pi}\wick{\psi\overset{\leftrightarrow}{\partial_z}
 \psi}(z)
 \]
 with central charge $c=1/2$. Commutation relations follow
 by implementing its decomposition in terms of Virasoro generators 
 \[
 T(z)=\frac{1}{2\pi}\sum_{n\in\Z}L_n z^{-n-2}
 \]
 where $L_n$ commute as in \eqref{Vir}. We obtain, on the
 real line and on the circle picture respectively:
 \begin{align*}
 \comm*{T(x),T(y)}&=i\left(T(x)+T(y)\right)\delta'(x-y)
 -i\,\frac{c}{24\pi}\,\delta'''(x-y)\\[2ex]
 \comm*{T(z),T(w)}&=\frac{-1}{2\pi}\left(\frac{T(z)}{w^2}+
 \frac{T(w^2)}{z^2}\right)\sum_{n\in\Z}n\left(\frac{z}{w}\right)^n\\
 &-\frac{c}{48\pi^2}\,\frac{1}{z^2 w^2}\sum_{n\in\Z}(n^3-n)
 \left(\frac{z}{w}\right)^n.
 \end{align*}
 In case of a complex Fermi field $\phi(z)=\psi^{(1)}(z)
 +i\psi^{(2)}(z)$ the related stress-energy tensor is
 exactly two copies of the individual stress-energy 
 tensors constructed out of each of the two real Fermi fields
 $\psi^{(1)}(z)$ and $\psi^{(2)}(z)$
 \[
 T^{c=1}(z)=\frac{-1}{4\pi}\,\wick{\psi^{(1)}\partial_z \psi^{(1)}}(z)
 +\frac{-1}{4\pi}\,\wick{\psi^{(2)}\partial_z \psi^{(2)}}(z)=
 \frac{-1}{8\pi}\wick{\phi^*\overset{\leftrightarrow}{\partial_z}
 \phi}(z)
 \]
 The central charge is $2\cdot1/2=1$. Let us now define again
 \begin{align*}
 T^{2c}(z)&\coloneqq T^{c}(z)\otimes\bm{1}+\bm{1}\otimes T^{c}(z)\\
 D^{2c}(z)&\coloneqq T^{c}(z)\otimes\bm{1}-\bm{1}\otimes T^{c}(z)
 \end{align*}
 and likewise the assignment
 \begin{align}
 \alpha\left(T^{2c}(z^2)\right)&\coloneqq
 \frac{T^c(z)+T^c(-z)}{4z^2}+\frac{c}{32\pi z^4}\label{T2c}\\[2ex]
 \alpha\left(D^{2c}(z^2)\right)&\coloneqq
 \frac{T^c(z)-T^c(-z)}{4z^2}\label{D2c}
 \end{align}
 which implies, for the Virasoro modes
 \begin{align*}
 \alpha\left(L_n^{2c}\right)&=\frac{1}{2}L_{2n}^c+\frac{c}{16}\,
 \delta_{n,0}\qquad n\in\Z\\[2ex]
 \alpha\left(D_{\nu}^{2c}\right)&=\frac{1}{2}L_{2\nu}^c
 \qquad \nu\in\Z+1/2.
 \end{align*}
 Virasoro relations for the generators $L_n^c$ imply
 \begin{align*}
 \comm*{\alpha\left(L_m^{2c}\right),\alpha\left(L_n^{2c}\right)}&
 =(m-n)\alpha\left(L_{m+n}^{2c}\right)+\frac{2c}{12}(m^3-m)\,
 \delta_{m+n,0}\\
 \comm*{\alpha\left(L_m^{2c}\right),\alpha\left(D_{\nu}^{2c}\right)}&
 =(m-\nu)\alpha\left(D_{m+\nu}^{2c}\right)\\
 \comm*{\alpha\left(D_{\mu}^{2c}\right),\alpha\left(D_{\nu}^{2c}\right)}&
 =(\mu-\nu)\alpha\left(D_{\mu+\nu}^{2c}\right)+\frac{2c}{12}(\mu^3-\mu)\,
 \delta_{\mu+\nu,0}
 \end{align*}
 summing up the Fourier modes we obtain 
 \begin{align*}
 \comm*{\alpha\left(T^{2c}(z^2)\right),\alpha\left(T^{2c}(w^2)\right)}&=
 \alpha\comm*{\left(T^{2c}(z^2)\right),\left(T^{2c}(w^2)\right)}\\
 \comm*{\alpha\left(T^{2c}(z^2)\right),\alpha\left(D^{2c}(w^2)\right)}&=
 \alpha\comm*{\left(T^{2c}(z^2)\right),\left(D^{2c}(w^2)\right)}\\
 \comm*{\alpha\left(D^{2c}(z^2)\right),\alpha\left(D^{2c}(w^2)\right)}&=
 \alpha\comm*{\left(D^{2c}(z^2)\right),\left(D^{2c}(w^2)\right)}
 \end{align*}
 meaning that $\alpha$, again, preserves the commutation relations. We 
 have thus embedded two copies of the stress-energy tensor algebra
 of central charge $c=1/2$ localised in $\I$ into one copy
 of the same algebra localised in $\sqrt{\I}$. Denoting such algebra
 as $\textrm{Vir}_{1/2}(\I)$ we have
 \[
 \alpha\colon\textrm{Vir}_{1/2}(\I)\otimes\textrm{Vir}_{1/2}(\I)
 \subset \textrm{Vir}_{1}(\I)\to\textrm{Vir}_{1/2}(\sqrt{\I}).
 \]
 Notice that the assignment $\alpha$ turns out to be nothing
 but the composition of $\beta$ with an automorphism of the current
 algebra $\rho^{\frac{1}{4}}$ as in \cite{Rehren:2012wa} 
 with $\rho^q(j(z))=j(z)+q/z$. Thus we have the 
 identification $\alpha=\beta\circ\rho^{\frac{1}{4}}$ and $\rho$ 
 plays the role of the gauge transformation $\gamma$ we have to compose 
 $\beta$ with in order to obtain suitable homomorphisms. 
 As a consequence, though $\beta$ itself does not restric to
 subalgebras, compositions with suitable inner automorphisms do 
 and we can extend the picture to fields of conformal scaling
 dimension $2$
 \begin{align*}
 {\text{Fermi}}^{\ d=\frac{1}{2}}& \qquad
             \begin{matrix}
             \phi(z^2)\\[1ex]
             \phi^*(z^2)
             \end{matrix}\
             \overset{\beta}{\longrightarrow}\
             \begin{matrix}
             \psi(z)+\psi(-z)\\[1ex]
             z^{-1}\left(\psi(z)-\psi(-z)\right)
             \end{matrix}\\[3ex]
 {\text{Currents}}^{\ d=1}&\qquad
             \begin{matrix}
             J_0(z^2)\\[1ex]
             J_3(z^2)
             \end{matrix}\
             \xrightarrow[]{\beta\,\circ\,\gamma}
             \begin{matrix}\
             z^{-1}\left(j(z)-j(-z)\right)\\[1ex]
             z^{-1}\left(j(z)+j(-z)\right)
             \end{matrix}\\[3ex]
 {\text{Virasoro}}^{\ d=2}&\qquad
             \begin{matrix}
             T^1(z^2)\\[1ex]
             D^1(z^2)
             \end{matrix}\
             \xrightarrow[]{\beta\,\circ\,\rho^{\frac{1}{4}}}
             \begin{matrix}\
             z^{-2}\left(T^{\frac{1}{2}}(z)+T^{\frac{1}{2}}(-z)\right)\\[1ex]
             z^{-2}\left(T^{\frac{1}{2}}(z)-T^{\frac{1}{2}}(-z)\right)
             \end{matrix}
 \end{align*}
 Again, we have a manifestation of the twist carried by $\beta$
 as flip on one of the two fields
 \begin{align*}
 \wp\left(T^1(z)\right)&=T^1(z),\\
 \wp\left({\Delta}^1(z)\right)&=-{\Delta}^1(z).
 \end{align*}

 
 \section{Coset models}
 \label{Coset models}
 The quarks construction as described in 
 \ref{The quarks construction}
 shows that starting from a Lie algebra $\mathfrak{g}
 \subset\mathfrak{u}(n)$
 one can construct a stress-energy tensor with a 
 suitable normalisation $\xi$ as 
 \[
 T_S(z)\coloneqq \xi\,\kappa_{ab}\,\wick{\,J^a J^b\,}(z)
 \]
 whose central charge is given by \eqref{coxeter} and whose
 Fourier modes satisfy the Virasoro algebra commutation
 relations. Let now take $\mathfrak{h}$ as a Lie subalgebra
 $\mathfrak{h}\subset\mathfrak{g}$ and let us apply the
 Sugawara construction to $\mathfrak{h}$ and $\mathfrak{g}$
 respectively. In general, the two stress-energy tensors,
 which we denote as $T^{\mathfrak{h}}, T^{\mathfrak{g}}$ 
 do not coincide. By taking $T^{\mathfrak{g}/\mathfrak{h}}(z)
 \coloneqq T^{\mathfrak{g}}(z)-T^{\mathfrak{h}}(z)$ (as shown in 
 \cite{GKO:1986},  \cite{GKO:1985}) we can construct
 another stress-energy tensor (``coset'' SET) 
 commuting with the $T^{\mathfrak{h}}$ 
 whose central charge is the difference of the
 central charges of the constituent models
 \[
 c^{\mathfrak{g}/\mathfrak{h}}=
 c^{\mathfrak{g}}-c^{\mathfrak{h}}.
 \]
 An important class of such models is given by the
 so called ``diagonal'' embeddings of an algebra into
 many of its commuting copies, $\mathfrak{g}
 \subset \mathfrak{g} \oplus\ldots\oplus\mathfrak{g}$. The total 
 current is then just the sum of each single current
 \[
 J^a(x)=\left(j^a(x)\otimes\bm{1}\otimes\ldots\bm{1}+
 \ldots+\bm{1}\otimes\ldots\otimes\bm{1}\otimes j^a(x)
 \right)
 \]
 and since the different copies commute with one other
 the level of the resulting algebra is just the sum
 of the level of each diagonal component. As a consequence
 the total central charge for the stress-energy tensor is
 \[
 c=\left(\frac{k_1}{k_1+g}+\ldots+\frac{k_n}{k_n+g}+
 \frac{k_1+\ldots+ k_n}{k_1+\ldots k_n+g}\right)
 \dm \mathfrak{g}.
 \]
 For example, taking $k$ copies of (level $k=1$) 
 $\mathfrak{su}(n)$ current models we can construct
 an $\mathfrak{su}(n)$ at level $k$ current model.
 Also, by iteration of this method one can get 
 representations of Virasoro algebras with $c<1$ just
 by taking diagonal embeddings of 
 ${\mathfrak{su}(n)}_{k+1}$ into ${\mathfrak{su}(n)}_k
 \oplus {\mathfrak{su}(n)}_1$, the total central
 charge being given by
 \[
 T^{{\mathfrak{su}(n)}_k\oplus {\mathfrak{su}(n)}_1/
 {\mathfrak{su}(n)}_{k+1}},\qquad
 c_k+c_1-c_{k+1}=1-\frac{6}{(k+2)(k+3)}.
 \]
 Similarly, the stress-energy tensor of two complex
 Fermi fields has central charge $c=2$, while the 
 stress-energy tensor for the embedded $\mathfrak{su}_1$
 has $c=1$; therefore one can construct a coset stress-energy 
 tensor whose central charge is $c=2-1=1$ and
 this happens to be exactly the abelian contribution
 $\mathfrak{u}(1)$ into $\mathfrak{u}(2)=\mathfrak{u}(1)
 \oplus \mathfrak{su}(2)$ of the form
 \[
 T(x)=\frac{1}{4\pi}\,\wick{j^2}(x)
 \]
 (see, for example, \cite{Fuchs:1992}).
 
 \section{Embedding via Longo-Xu map}
 \label{Embedding via Longo-Xu map}
 We have obtained the results previously mentioned by using the 
 embedding of $\beta$ and some gauge transformations
 suitably chosen. We shall now show that the same
 result can be easily achieved by using the general 
 transformation properties of conformal fields under 
 diffeomorphisms and the Longo-Xu map (\cite{LX:2004}
 and \ref{Longo-Xu}). As a recall we assume to be equipped
 with a conformal net $\I\to\alg{\I}$ fulfilling the 
 split property: therefore, given $\I_1,I_2$ 
 there exists an isomorphism $\chi\colon\alg{I_1}\vee
 \alg{I_2}\to\alg{I_1}\otimes\alg{I_2}$ such that
 $\chi(x_1 x_2)=x_1\otimes x_2$, however you choose
 $x_1\in \I_1,x_2\in\I_2$. Diffeomorphisms of the net
 $\mu\colon\I\to\I_j$ are implemented on the algebras
 by means of adjoint action of unitaries $U(\mu)$ and
 this provides isomorphisms $\alg{\I}\to\alg{\I_j}$ 
 given by (\cite{LX:2004})
 \[
 \phi_I\coloneqq\left.\ad U(\mu)\right|\alg{\I}.
 \]
 Consequently the Longo-Xu map 
 $\LX=\chi\circ\phi_{\I}^N$ gives an 
 isomorphism 
 \[
 \LX\colon\mathcal{A}^N(\I)\to\mathcal{A}
 (\I_1\cup\ldots \cup\I_N)
 \]
 explicitly realised as
 $\LX(x_1\otimes\ldots\otimes x_N)=\phi_{\I}(x_1)
 \cdot\ldots\cdot\phi_{\I}(x_N)$.
 
 \bigskip
 Let us now restrict ourselves to the particular case
 of two intervals and diffeomorphisms given in terms of 
 square root maps, namely $\mu_1(z^2)=z,\,\conj{\mu}(z^2)=-z$. 
 They are implemented on fields as 
 \[
 \phi_{\I}(\varphi(z))=\ad U(\sqrt{\blank})\varphi(z)=
 \left(\frac{\partial \mu(z)}{\partial z}\right)^h\,
 \varphi(\mu(z))
 \]
 $h$ being the conformal scaling dimension. The corresponding Longo-Xu
 map is $\LX\colon\mathcal{A}^2(\I)\to\alg{\sqrt{\I}}$. On the
 doubled currents 
 \begin{align*}
 J^{2c}(z^2)&=J^c(z^2)\otimes\bm{1}+\bm{1}\otimes J^c(z^2)\\
 \Delta^{2c}(z^2)&=J^c(z^2)\otimes\bm{1}-\bm{1}\otimes J^c(z^2)
 \end{align*}
 the Longo-Xu map acts as
 \begin{align*}
 \LX\left(J^{2c}(z^2)\right)&=\left(\frac{1}{2z}\right)^1 j^c(z)\bm{1}
 + \bm{1}\left(\frac{-1}{2z}\right)^1 j^c(-z)=
 \frac{j(z)-j(-z)}{2z}\\[2ex]
 \LX\left(\Delta^{2c}(z^2)\right)&=\left(\frac{1}{2z}\right)^1 j^c(z)\bm{1}
 - \bm{1}\left(\frac{-1}{2z}\right)^1 j^c(-z)=
 \frac{j(z)+j(-z)}{2z}
 \end{align*}
 which exactly correspond to equations \eqref{J2c} and \eqref{delta2c}.
 Consequently we derive that $\alpha=\beta\circ\gamma=\LX$ and
 thus $\beta$ and $\LX$ are related to each other through a gauge
 transformation $\gamma$. Of course this must be the case, since
 the Longo-Xu map does not preserve the vacuum state (diffeomorphisms
 ``destroy'' correlations), while $\beta$ does. 
 
 Similarly we can apply the Longo-Xu map to the stress-energy tensor
 and its doubled copy
 \begin{align*}
 T^{2c}(z)&\coloneqq T^{c}(z)\otimes\bm{1}+\bm{1}\otimes T^{c}(z)\\
 D^{2c}(z)&\coloneqq T^{c}(z)\otimes\bm{1}-\bm{1}\otimes T^{c}(z)
 \end{align*}
 keeping in mind that $T$ does not transform as a primary field
 under diffeomorphisms, rather it is quasi-primary and an extra 
 contribution due to the Schwarz derivative occurs.
 \begin{align*}
 \LX\left(T^{2c}(z^2)\right)&= 
                \left(\left(\frac{1}{2z}\right)^2 T(z)+\right.
                \left.\frac{c}{12}s(g(z),z)\right)\bm{1}\\
 &+ \bm{1}\left(\left(\frac{-1}{2z}\right)^2 T(-z) +\right.
                \left.\frac{c}{12}s(g(z),z)\right)\\
 &=\frac{T(z)+T(-z)}{4z^2}+\frac{c}{32\pi z^4}.
 \intertext{On the other hand the Schwarzian derivative
 cancels out if we take the difference}
 \LX\left(D^{2c}(z^2)\right)&= \frac{T(z)-T(-z)}{4z^2}.
 \end{align*}
 We have obtained equations \eqref{T2c} and \eqref{D2c}
 just via mere application of the diffeomorphisms invariance
 and the split property, which we assume to hold for the net 
 at hand. We deduce again $\alpha=\beta\circ\rho^{\frac{1}{4}}
 =\LX$. 

 \bigskip
 \begin{figure}[htbp]
 \centering
 \def\radius{1.05}
 \def\rad{0.95}
 \def\out{1.15}
 \begin{tikzpicture}[scale=1.5]
 % circle
 \path (0,0) coordinate (O);
 \draw [style=dashed](O) circle (\rad);
 % boundary points
 \path (20:\radius) coordinate (A2);
 \path (260:\radius) coordinate (B2);
 \path (140:\radius) coordinate (C2);
 %middle points
 \path (30:\out) coordinate (A3);
 \path (280:\out) coordinate (B3);
 \path (158:\out) coordinate (C3);
 \path (163:\out) coordinate (C4);
 % control points
 \path (90:1.3) coordinate (P1);
 \path (220:1.3) coordinate (P2);
 % arcs
 \draw [line width=0.4mm, color=navy!60!] (A2) arc (20:60:\radius);
 \draw [line width=0.4mm, color=navy!60!] (B2) arc (260:300:\radius);
 \draw [line width=0.4mm, color=navy!60!] (C2) arc (140:180:\radius);
 \draw plot [smooth, tension=3] coordinates {(C3) (P1) (A3)};
 \draw plot [smooth, tension=3] coordinates {(C4) (P2) (B3)};
 % nodes
 \node at (160:1.7) {$\phi(z^2)$};
 \node at (20:1.5) {$\phi(z)$};
 \node at (300:1.5) {$\phi(-z)$};
 \end{tikzpicture}
 \end{figure}
 The picture that we have now is that, for each scaling dimension,
 i.e. for Fermi fields, embedded currents and stress-energy tensor,
 although $\beta$ does not exactly restrict to the respective subalgebra,
 its composition with suitable gauge transformations gives back exactly
 the Longo-Xu map, which in turn is the manifestation of the
 diffeomorphisms covariance of the net (assumed the split property
 to hold). Both maps, $\beta$ and $\LX$, somehow ``distribute'' fields
 around the circle $\I\to\sqrt{\I}$ and they are related to each other via 
 tailor made gauge transformations:
 \[
 \beta=\LX\circ\text{gauge}
 \]
 where these acquires the explicit forms
 \begin{align*}
 \text{Fermi fields:}&\qquad\LX^{d=\frac{1}{2}}=
 \beta\circ O^{\text{CH}}\\
 \text{Currents:}&\qquad\LX^{d=1}=\beta|_J\circ\gamma\\
 \text{Stress-energy tensor:}&\qquad\LX^{d=2}=
 \beta|_{\text{Vir}}\circ\rho^{\frac{1}{4}}
 \end{align*}
 (of course we can read off $\gamma=O^{\text{CH}}|_J$ and
 likewise $\rho^{\frac{1}{4}}=O^{\text{CH}}|_{\text{Vir}}$).
 
 \section{Modular theory for currents}
 \label{Modular theory for currents}
 In the previous paragraphs we introduced the doubled 
 theory of currents $\mathcal{A}^J\otimes\mathcal{A}^J$
 as embedded from fermions using the 
 quark construction: given a theory $\mathcal{A}^2(\I)$
 describing two fermions in one interval we can generate
 the embedded theory of currents $\mathcal{A}^J(\I)
 \hookrightarrow\mathcal{A}^2(\I)$. 
 Furthermore, we found out that the 
 restriction of $\beta|_J$ can still be written as 
 $\beta|_J=\LX|_{\textrm{F}}\circ\gamma$, where 
 $\LX|_{\textrm{F}}$ denotes the restriction of the 
 Longo-Xu map to the embedded fermions. Of course, 
 since $\beta$ preserves the vacuum state for Fermions,
 so does it restriction to currents. Nonetheless, one may 
 wonder how the very particular form for gauge transformations
 on currents, equation \eqref{GTcurrents}, may fit so that
 the correlators are eventually preserved. This is due to 
 the particular action of $\beta$ on Fermi fields; in fact
 its peculiarity is to distribute fields onto anti-podal 
 points and this feature reflects on the currents, as 
 in formula \eqref{multi-local_fermionisation}. The 
 presence of delocalised factors in, say, $z,-z$ exactly
 cancels out the additional central term appearing in the 
 gauge transformations so that everything cancels out 
 eventually preserving the form of the vacuum expectation
 values. As a matter of example we shall present the case 
 of a doubled theory of currents originated from four 
 fermions.
 \begin{example}
 Let $\mathcal{A}^2(\I)$ be the theory describing two fermions,
 say $\psi_1,\psi_2$ and $\mathcal{A}^2(\I)\otimes
 \mathcal{A}^2(\I)$ its double. The four fermions generated 
 thereof can be labelled as 
 \[
 \Psi^1=\psi_1\otimes\bm{1},\quad \Psi^2=\psi_2\otimes\bm{1},
 \quad \Psi^3=\bm{1}\otimes\psi_1,\quad
 \Psi^4=\bm{1}\otimes\psi_2.
 \]
 These four fermions can generate $\binom{4}{2}=6$ different 
 currents $J_{ij}(z)\coloneqq \Psi^i(z)\,\Psi^j(z)\in
 \mathcal{A}^J(\I)\otimes\mathcal{A}^J(\I)$ (no Wick product
 occurs because the vacuum expectation values vanish anyway) 
 generating in turn a non-abelian current algebra 
 with gauge group $\textrm{O}(4)$. We shall see that the 
 action of the Longo-Xu map on these currents is local 
 on some pairing, whereas it is non-local on some 
 others. In fact, let us take the action of 
 diffeomorphisms $\mu_j\colon \I\to\I_j$ as 
 $\mu_1(z)=\sqrt{z},\mu_2(z)=-\sqrt{z}$ upon, 
 for instance,
 $J_{12}(z)=\Psi^1(z)\Psi^2(z)=
 \psi_1(z)\psi_2(z)\otimes\bm{1}$; we have 
 \begin{multline*}
 \LX(J_{12}(z))=\LX(\psi_1(z)\psi_2(z)\otimes\bm{1})\\=
 \sqrt{\mu_1'(z)}\,\psi_1(\mu_1(z))\,
 \sqrt{\mu_1'(z)}\,\psi_2(\mu_1(z))\cdot\bm{1}
 \end{multline*}
 because the diffeomorphisms both act on the first 
 term in the tensor product, distributing the fields
 in $\mu_1(z)$. Then
 \[
 \LX(J_{12}(z))=\mu_1'(z)\,\psi_1(\mu_1(z))\,
 \psi_2(\mu_1(z))=\mu_1'(z)\,\wick{\psi_1\psi_2}(\mu_1(z))
 \]
 exploiting $\wick{\psi_1\psi_2}=\psi_1\psi_2-\vac(
 \psi_1\psi_2)=\psi_1\psi_2-0$. We can consequently 
 state that the current $J_{12}$ is distributed 
 locally at the point $\mu_1(z)$; the same happens
 for those other currents having the initial fermions
 in the same position in the tensor product, like, for 
 example, $J_{34}(z)=\Psi^3(z)\Psi^4(z)=\bm{1}\otimes 
 \psi_1(z)\psi_2(z)$
 \[
 \LX(J_{34}(z))=\mu_2'(z)\,
 \wick{\psi_1\psi_2}(\mu_2(z));
 \]
 we conclude then that the two possible local actions 
 are the following ones:
 \[
 \LX(J_{12}(z))=\mu_1'(z)\,j_{12}(\mu_1(z)),\qquad 
 \LX(J_{34}(z))=\mu_2'(z)\,j_{12}(\mu_2(z)).
 \]
 The other possible pairings present non-local 
 contributions, as well as, for example, 
 $J_{13}(z)=\Psi^1(z)\Psi^3(z)=\psi_1(z)\otimes^t 
 \psi_1(z)$
 \[
 \LX(J_{13}(z))=\sqrt{\mu_1'(z)}\,\psi_1(\mu_1(z))
 \,\sqrt{\mu_2'(z)}\,\psi_1(\mu_2(z));
 \]
 because of the presence of the same fermion field 
 $\psi_1$, the vacuum expectation value is non-zero 
 and therefore we get
 \begin{multline*}
 \LX(J_{13}(z))=\sqrt{\mu_1'(z)\mu_2'(z)}\,\psi_1(\mu_1(z))
 \,\psi_1(\mu_2(z))\\
 =\sqrt{\mu_1'(z)\mu_2'(z)}\,\wick{\psi_1(\mu_1(z))\,
 \psi_1(\mu_2(z))}-\frac{\sqrt{\mu_1'(z)\mu_2'(z)}}
 {\mu_1(z)-\mu_2(z)}
 \end{multline*}
 which is delocalised in the two points 
 $\mu_1(z),\mu_2(z)$.
 
 \bigskip
 The action of the diffeomorphisms is a ``true'' 
 action only on some of the currents, taking them  
 into actual currents localised elsewhere. This can be 
 viewed as a true action on the currents of the 
 subgroup $\textrm{O}(2)
 \times\textrm{O}(2)\subset\textrm{O}(4)$, while 
 the remaining currents are moved in $\mu_1(z),
 \mu_2(z)$ without summing up again to actual currents.
 
 We can use this argument to reconstruct the form of 
 the one-point function. In fact, since $\vac(J(z))=0$,
 we expect $\beta|_J$ to preserve $\vac\circ\beta|_J=
 \vac$. Acting on currents we obtain 
 \begin{align*}
 \vac\circ\beta|_J(J(z))&=\vac\circ\LX\circ\gamma(J(z))\\
 &=\vac\circ\LX(J(z)+\textrm{central term})
 \end{align*}
 where $(J(z)+\textrm{central term})$ has to be intended as
 in equation \eqref{GTcurrents} and the central term 
 is of the form $1/2\,
 \kappa\,\tr\left(g(z)^{-1}\partial_z g(z)\sigma^a\right)$.
 Using the explicit form of the Lie algebra valued gauge 
 transformations and the structure constants the trace sums
 up to either zero or the identity, the only numerical 
 prefactors being derivatives of the diffeomorphisms $\mu$
 in the point $z$, which cancel the presence of the 
 additional vacuum expectation value in some delocalised
 currents appearing in the model once we act with the
 Longo-Xu map. Again, the multi-local behaviour of 
 $\beta$ helps to prevent obstructions and to preserve 
 the one-point function:
 \begin{align*}
 \vac\circ\beta|_J(J(z))&=\vac
 \circ\LX(J(z)+\textrm{central term})\\
 &=\vac\Bigg(J(z) \under{\textrm{cancellations}}
 {-\frac{\sqrt{\mu_1'(z)\mu_2'(z)}}
  {\mu_1(z)-\mu_2(z)}+ \textrm{central term}}\Bigg)\\
  &=\vac(J(z))=0.
 \end{align*}
 \end{example}
 On the other hand, if the action were strictly local on 
 all the involved currents, then obstructions for the 
 vacuum one-point function would definitely occur,
 because cancellations would no more take place and 
 the additional central term arising from the gauge 
 transformations could not be wiped off. As a consequence,
 if we start from a pure local theory of currents, no 
 vacuum preserving isomorphism in the form 
 $\beta=\LX\circ\gamma$ may exist.
 As distribution $\vac(j(x))=0$  means that
 there can be no class of functions $f$ such that 
 $\vac(j(f))\neq 0$. From this it directly follows that
 a similar argument applies to the non-existence of 
 a vacuum preserving isomorphism for Bose fields; in 
 fact, if this were true then we would have
 \[
 \vac\circ\beta\left(\varphi(f)\right)=
 \vac\left(\varphi(f)\right)
 \]
 choosing $f$ integrating to zero. Then, since 
 $\varphi(f)=-j(g)$, with $f(x)=g'(x)$ then this would
 imply
 \[
 \vac\circ\beta\left(j(g)\right)=
 \vac\left(j(g)\right)=0
 \]
 and this cannot hold due to the explicit 
 form of $\beta=\LX\circ\textrm{gauge}$. In fact, as we 
 pointed out, $\beta(j)\sim j + \textrm{const.}\cdot\bm{1}$
 and thus
 \[
 \vac\circ\beta\left(j(g)\right)=
 \vac(j(g))+\int_{\R}\dd x\,g(x)=
 0 +\int_{\R}\dd x\,g(x)
 \]
 and this is again if also $g(x)$ integrates itself to
 zero, but this cannot go along with the integral of
 $f(x)=g'(x)$ being zero as well.
 \begin{example}
 Let us take again the doubled theory of two fermions,
 $\mathcal{A}^2(\I)\otimes\mathcal{A}^2(\I)$, containing
 the four fermions as stated in the previous example 
 \[
 \Psi^1=\psi_1\otimes\bm{1},\quad \Psi^2=\psi_2\otimes\bm{1},
 \quad \Psi^3=\bm{1}\otimes\psi_1,\quad
 \Psi^4=\bm{1}\otimes\psi_2.
 \]
 The corresponding stress-energy tensor is, by 
 construction,
 \[
 T(z)=-\frac{1}{4\pi}\sum_{i=1}^4\,
 \wick{\Psi^i\partial_z\Psi^i}(z)
 \]
 and thus 
 \begin{multline*}
 T^{2c}(z)=-\frac{1}{4\pi}\wick{\psi_1\partial_z\psi_1}(z)\otimes\bm{1}
 -\frac{1}{4\pi}\wick{\psi_2\partial_z\psi_2}(z)\otimes\bm{1}\\
 +\bm{1}\otimes-\frac{1}{4\pi}\wick{\psi_1\partial_z\psi_1}(z)
 +\bm{1}\otimes-\frac{1}{4\pi}\wick{\psi_2\partial_z\psi_2}(z),
 \end{multline*}
 which is nothing but $T^{2c}(z)=T^c(z)\otimes\bm{1}
 +\bm{1}\otimes T^c(z)$. Due to the fact that there are no 
 mixed terms paired in the tensor products, the action of the 
 $\LX$ map is local on each component of the tensor product
 individually. We have 
 \begin{multline*}
 \LX\big(T^{2c}(z)\big)=\LX\big(T^c(z)\otimes\bm{1}
 +\bm{1}\otimes T^c(z)\big)\\
 =\Big(\mu_1'(z)^2\,T^c(\mu_1(z))+\frac{c}{12}s(\mu_1(z),z)\Big)
 \cdot\bm{1}\\
 +\bm{1}\cdot\Big(\mu_2'(z)\,T^c(\mu_2(z))+
 \frac{c}{12}s(\mu_2(z),z)\Big).
 \end{multline*}
 On the other hand if one considers the Sugawara
 stress-energy tensor $T_s(z)=\xi\,\kappa_{ab}
 \wick{J^aJ^b}(z)$ with currents given by $J_{ij}(z)
 =\Psi^i(z)\Psi^j(z)$ then anti-local components 
 may occur, according to the choice of the Lie
 algebra.
 \end{example}
