%*****************************************
\section{Diffeomorphism covariance versus multi-locality}
\label{connections with gauge}
%*****************************************
So far we have seen that there are essentially two 
distinguished ways to distribute fields around the circle. The
first one is by implementing diffeomorphisms 
covariance, if the net is assumed to fulfill such
property; so to speak, under a general change 
$z\mapsto \mu(z)$ fields change accordingly as 
$\phi'(\mu(z))=(\partial \mu/\partial z)^h\phi(\mu(z))$,
where $h$ is the field scaling dimension. Unitary
operators causing such displacement do exist and
thus $\phi'(\mu(z))=\ad (U(\mu))(\phi(z))$. This
concept has led \cite{LX:2004} to the introduction
of the Longo-Xu map as isomorphism of algebras
\eqref{L-X}. Because of the split property being involved, 
the Longo-Xu map does not preserve the vacuum state,
because correlations are explicitly broken into 
product of states. This behaviour does not reflect 
quantum field theory at all, whose 
principal feature is that fields and observables 
must be correlated anyway, affecting each other 
according to the Einstein causality principle.

\bigskip
On the other hand, after investigation of modular 
theory for fermions in disjoint intervals, we have 
found that, however, an isomorphism of algebras 
preserving the vacuum state does exist in the form 
given by equation \eqref{map_beta_general}. 
Albeit in principle the two concepts might seem 
unrelated, they happen to
be intimately connected through suitable gauge 
transformations. We start showing a simple example 
thereof and then we move to the general case.
\begin{example}
Let us take as diffeomorphisms the square root 
map $z\mapsto \pm\sqrt{z}$. The action of the 
corresponding Longo-Xu isomorphism 
on the complex fermion looks like equation
\eqref{LX_complex}
\[
\LX(\phi(z^2))=\frac{1}{2\sqrt{z}}
\left(\psi(z)+\psi(-z)\right).
\]
We can compare this formula with \eqref{beta_complex}, which
shows the same action under the map $\beta$, respectively:
\[
\beta(\phi(z^2))=\frac{1}{2}
\left(\psi(z)+\psi(-z)\right).
\]
Remarkably we see that the two actions are related as 
\[ 
\LX(\phi(z^2))=(z^2)^{-1/4}\,
\beta(\phi(z^2)),
\]
which can be rewritten as $\beta=\LX\circ \gamma$,
where $\gamma\colon\mathcal{A}^2(\I)\to\mathcal{A}^2(\I)$
acts on the complex fermion as $\gamma(\phi(z))=z^{1/4}
\phi(z)$ and can be interpreted as a
gauge transformation, in particular like pointlike 
rotations $\rot(\varphi/4)$
if $z=\e^{i\varphi}$.
\end{example}
This is not accidental, rather it is pretty general.
Let us work for the sake of simplicity on the real line,
making use of the uniformisation funcion $X(x)$ as variable
at hand. Gauge transformations $\gamma\colon\alg{\I}\to\alg{\I}$
preserving each subalgebra $\alg{\I}$ can be defined 
by their action on $\mathcal{A}^N(\I)$ as
\begin{equation}
\label{gamma_gauge}
\gamma(\psi_i)=\sum^N_{r=1}O(X)_{ir}\,\psi_r(X)
\end{equation}
with $O(X)$ being the mixing matrix appearing in the
Casini-Huerta modular flow for disjoint intervals. 
As such, $O(X)$ can be read as an $\textrm{SO}(N)$ 
valued function and $\gamma$ takes fields at the point
$X$ into combination of fields at the same point, with 
pointlike dependent coefficient (which justifies the 
name of gauge transformations). Choosing now the 
funcion $X(x)$ as diffeomorphism to be implemented, 
we can act with the Longo-Xu map upon
\eqref{gamma_gauge} in order to have 
\begin{align*}
\LX\circ\gamma(\psi_i(X))&=\LX\Big(\sum^N_{r=1}O(X)_{ir}
\psi_r(X)\Big)\\
&=\sum^N_{r=1}O(X)_{ir}\sqrt{X_r^{-1}(X)'}
\,\psi(X_r^{-1}(X))
\end{align*}
that is $\beta(\psi_i(X))$, however we choose $\psi_i$
as element of the algebras, and thus 
\begin{equation}
\label{beta_versus_LX}
\beta=\LX\circ\gamma.
\end{equation}
The vacuum preserving isomorphism $\beta$ is related to the 
Longo-Xu map via choice of suitable gauge transformations,
expressed by the orthogonal cocycle $O(X)$. Composition
with such a map exactly \emph{undoes} the split property and 
brings back the correlations among fields. In fact,
although neither $\LX$ nor $O(X)$ preserve the 
vacuum state by themselves, their joint action does,
bringing back the exact form of $\beta$ as appearing
in the modular flow for free fermions in disjoint intervals.
We shall see later on, in the following chapters about 
currents, that this is a general issue which will be
present in the embedded models as well. In fact, 
even though $\beta$ itself does not completely restrict 
to the embedded subalgebras, its composition with $\gamma^{-1}$ 
does, giving back the Longo-Xu map for currents and 
stress-energy tensor in the context of doubled theories;
however, concerning these topics, we refer the reader
to forthcoming chapters 
\ref{Currents models} and \ref{Embedding via Longo-Xu map}.

















%*****************************************
%*****************************************
%*****************************************
%*****************************************
%*****************************************