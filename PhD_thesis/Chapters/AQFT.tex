%************************************************
\chapter{Introduction to quantum field theory}\label{ch:QFT}
\minitoc\mtcskip
%************************************************

\noindent A rigorous inspection of the behaviour of quantum field theory
showed some common general features which were seen to be 
always realised, no matter the physical system at hand.
Such features have been thus taken as defining properties 
(axioms) of the quantum field theory itself and the study of their 
mathematical properties leads to the characterisation 
of algebraic quantum field theory. We shall introduce such
postulates following the example given by the standard
textbook in this area, namely \cite*{Haag}.

\bigskip
The main objects any physical theory deals with, no matter
whether classical or quantised, are fields $x\mapsto\phi(x)$
(whose mathematical properties have to fulfill the requirements 
of the model at hand).
Their role is to implement the principle of locality; 
\emph{observables} are the quantities that can be directly
reproduced in a laboratory and they can in general be
read off and reconstructed once the field content is assigned.
Fields themselves may also be observables, though they need not to.

%*******************************************************
\section{General postulates: Wightman axioms}
\label{Wightman}
 \begin{enumerate}[label=\Alph*.]
  \item \emph{Fields}:        
        Fields are operator valued distributions on Minkowski space.
        This means that the linear assignment $f\mapsto\phi(f)$ gives
        back an (usually unbounded) operator on some Hilbert space 
        $\hil$ with dense domain $\mathfrak{D}(\phi(f))\subseteq\hil$. 
        The assignment has to be thought as a smearing
        \[
        \phi(f)=\int_{\vN} \dd^4x\,\phi(x)f(x)
        \]
        with $f$ belonging to some suitable functional space $\mathcal{F}$.
        The further assumption $\phi(f)\mathfrak{D}\subset\mathfrak{D}$
        ensures that we may operate arbitrarily many times with fields
        upon vectors $\in\mathfrak{D}$.
  \item \emph{Poincar\'{e} group and transformation properties}: 
        The Hilbert space $\hil$ carries
        a unitary representation $U(g)$ of the covering of the 
        Poincar\'{e} group $\poin$. The spectrum of the energy-momentum
        operators $P^{\mu}$ is contained in the forward light cone
        and this ensures consistency with special relativity,
        $p^2\coloneqq m^2\geq 0, p^0\geq 0$. Moreover, let $\loren\subset\poin = 
        \R^4 \rtimes \loren$ be the Lorentz subgroup of the Poincar\'{e}
        group and let $U(\Lambda,a)$ be a representation of $\poin$
        with $\Lambda\in\loren,\,a\in\R^4$. Fields transform under $\poin$~as
        \[
        U(\Lambda,a)\,(\phi(x))\,U^*(\Lambda,a)= S(\Lambda^{-1})\,
        \phi(\Lambda x+a),\quad S(\Lambda^{-1})\in\,\loren.
        \]
        In a nutshell the choice of $S(\Lambda)$ characterises the
        ``spin'' of the field.
  \item \emph{Hermiticity}: Given a field $\phi(f)$, the theory
	contains also the hermitian conjugate field $\phi(f)^*$ 
	defined so that 
	\[
	\scal{\Phi,\phi(f)\Psi}=\scal{\phi^{\dagger}(\overline{f})\Phi,\Psi}.
	\]
	Fields may be self-adjoint, $\phi(x)=\phi^{\dagger}(x)$ and
	thus $\scal{\Phi,\phi(f)\Psi}=\scal{\phi(\overline{f})\Phi,\Psi}$
	given $\Phi,\Psi$.
  \item \emph{Locality}: If the supports of the test functions $f$ and
	$g$ are space-like to each other, then fields must satisfy either
	of the following commutation relations
	\[
	\comm{\phi(f),\phi(g)}\Psi=0\quad\text{or}\quad
	\ant{\phi(f),\phi(g)}\Psi=0,\quad\Psi\in\mathfrak{D}. 
	\]
        Fields of the former type are called ``bosonic'', whereas
        fields of the latter type are called ``fermionic''. Due
        to Einstein causality observables must commute at space-like
        distances, therefore fermionic fields by themselves cannot be
        observables, whilst bosonic fields may.  
  \item \emph{Vacuum state and completeness}: There exists a unique state 
        $\Omega\in\hil$ invariant under 
        $U(g),\,g\in\poin$. Such a state is referred to as the 
        ``vacuum state''. Also, by acting upon the vacuum with an 
        arbitrary polynomial in the fields $\phi(f)$ one can approximate 
        any operator acting on $\hil$.        
 \end{enumerate}
It turns out that these properties are easily realised by free fields
satisfying linear equations, while constructions in terms of 
interacting fields are very difficult to achieve.

 \begin{definition}
 Let $\Omega$ be the vacuum vector. The vacuum expectation values
 \[
 w^{(n)}(x_1,\ldots,x_n)\coloneqq
 \scal{\Omega,\phi(x_1)\ldots\phi(x_n)\Omega}
 \]
 are called (Wightman) $n$-points correlation functions, though
 they are, more precisely, tempered distributions on $\R^{4n}$.
 \end{definition}
A fundamental result in this respect is the ``reconstruction theorem''
\cite*{Haag}, namely, under some suitable assumptions that we do not 
discuss in here, the whole fields content can be derived out of the 
knowledge of all correlation functions. 

\section{Fermi fields versus Bose fields}
\label{Fermi vs Bose}
As previously stated, fields appearing in nature must satisfy particular
restrictions on the way they commute between each other, this being
express by either commutation or anti-commutation relations. 
Fields of the former kind are referred to as ``Bose fields''
whereas fields of the latter kind are usually referred to as 
``Fermi fields''. In particular, those fields that belong 
to integer ``spin representations'' of the Lorentz group 
(in the sense of $S(\Lambda)$, as we have seen 
before) are Bose fields, while those ones that belong to
half-odd integer representations are Fermi fields. Such
particular feature characterises the spin-statistic
theorem (\cite{Haag}). As a first remark notice
that Bose fields might in principle be already observables, 
because they automatically fulfill Einstein causality; 
on the other hand Fermi fields do not, and observables 
must be constructed as particular combinations
of them (currents and stress-energy tensor, as we will 
show later on). However, we shall show the explicit 
construction of operator algebras based on the above 
commutation relations in the very special case when 
the space-time is one-dimensional, where this has
to be understood as previously mentioned, namely 
as decomposition in terms of light-ray variables.

\bigskip
Let us construct fermionic fields first. Take $\hil$ as 
any Hilbert space of functions with an involution 
$\Gamma\mid(\Gamma f)(x)=\conj{f(x)}$. Through the 
following linear assignment $f\mapsto\psi(f)$, which 
can be thought as an integral smearing, we can construct 
the set
\[
\textrm{CAR}(\hil,\Gamma)\coloneqq
\conj{\set{\psi(f)\mid f\in \mathcal{\hil},\ 
(\Gamma f)(x)=\conj{f(x)}}}_{\norm{\blank}}\,.
\]
The norm of an operator in such a set is uniquely fixed by the relation
$\psi(f)^*=\psi(\Gamma f)$ and by the anti-commutation relations
\begin{equation}
\label{CAR}
\ant{\psi(f)^*,\psi(g)}=\scal{\Gamma f,g}_{\hil}\,\bm{1};\quad
f,g\in\hil
\end{equation}
According to the choice of the Hilbert space one can realise either
real fields or complex fields. The standard choice is to take
$\hil = \LL{\R,\dd x}$ to have real fields and two such copies
$\LL{\R}\oplus\LL{\R} = \LL{\R}\otimes\C$ to have complex fields.
The norm is then seen to satisfy the inequality $\norm{\psi(f)}
\leq \norm{f}_{\hil}$ and therefore the operators are
bounded by the norm of the functions in $\hil$. By choosing
a projection $P\mid \Gamma P\Gamma=\bm{1}-P$ one can decompose
fields into creation and annihilation modes
$\psi(f)=\psi\left(Pf + \Gamma P\Gamma f\right)=
\psi(Pf)+\psi(\Gamma P\Gamma f)=\psi^+(f)+\psi^-(f)$
and also define two-point function as
\[
\omega_{P}\left(\psi(f)\psi(g)\right)\coloneqq
\scal{\Gamma f, P g}_{\hil}.
\]
Positivity is ensured by positivity in the Hilbert space
and higher order correlation functions can be defined 
\cite{Boeck:1996} as
\begin{multline}
\label{wick}
\omega_P\left(\psi(f_1)\ldots\psi(f_{2n})\right)\coloneqq\\
{(-1)}^{1/2\,n(n-1)}\sum_{\sigma\in \textrm{P}_n}\textrm{sign}\,
\sigma\,\prod^n_{j=1}\omega_P
\left(\psi(f_{\sigma(j)})\psi(f_{\sigma(n+j)})\right)
\end{multline}
with all the odd correlation functions vanishing. Such a state is
usually called quasi-free. The corresponding irreducible GNS 
representation gives the state in terms of scalar product as
expressed in the previous paragraph.
\begin{example}[Real Fermi field]
The real Fermi field on the real line
can be decomposed into Fourier modes as
\[
\psi(x)=\frac{1}{\sqrt{2\pi}}\int_{\R}\dd k\,a(k)\e^{-ikx}
\]
with the reality condition $a^*(k)=a(-k)$ and anti-commutation
relations $\ant{a(k),a(k')^*}=\delta(k-k')$. 
At the level of distributions, commutation relations for the 
fields themselves are 
\[
\ant{\psi(x),\psi(y)}=\delta(x-y),\quad x,y\,\in\R.
\]
Taking into account
that $a(k)$ annihilates the vacuum for each $k$, the one point
function is easily seen to vanish, $\vac(\psi(x))=0$, whereas  
the vacuum two-point function is 
\[
\vac(\psi(x)\psi(y))=\int_{\R} \dd k \e^{-ikx} \int_{\R}\dd k'
\e^{-ik'y} \vac(a(k)a(k'))
\]
which becomes, after using the anti-commutation relations
for the Fourier modes
\begin{multline}
\label{vacuum2}
\vac(\psi(x)\psi(y))=\int_0^{\infty}\dd k\,\e^{-ik(x-y)}=\\
\lim_{\varepsilon\to 0^+}\int_0^{\infty}\dd k\,
\e^{-ik(x-y)-k\varepsilon}
=\lim_{\varepsilon\to 0^+}\frac{-i}{x-y-i\varepsilon}
\end{multline}
and we shall encounter this formula many times later on 
(e.g \ref{repn of Fermi on the circle}). The projection 
defining the two-point function is the projection 
onto the positive modes, $P=\chi\big(\open{0}{\infty}\big)$
such that 
\[
(Pf)(x)=\int_0^{\infty}\dd k\, \tilde{f}(k)\,\e^{-ikx}.
\]
\end{example}

\bigskip
The construction of Bose field works similarly, with the exception
that commutation relations pose some obstructions for the norm
of the operators to be bounded. However one starts from the 
assignment $f\mapsto \phi(f)$ and defines the Weyl operators
as the exponential $W(f)=\e^{i\phi(f)}$. Commutation relations
are then implemented by means of a skew-symmetric two form
$\sigma\colon (f,g)\mapsto\sigma(f,g)$ as
\[
W(f)W(g)=\e^{i/2\,\sigma(f,g)}W(f+g).
\]
The set of all $W(f)$ is a *-algebra and imposing the condition
$\norm{W(f)}=1$ ensures that it has a unique $C^*$ norm. The set
\[
\conj{\set{W(f)\mid f\in \hil}}_{\norm{\blank}}
\eqqcolon \text{CCR}(\hil,\sigma)
\]
is then turned into a $C^*$-algebra. Notice in turn that unitarity 
and the Weyl commutation relations imply $W(0)=1$ and $W(f)^*=
W(-f)$. Along the same lines as before, representations may emerge
assigning the state $\omega(W(f))=\e^{-1/2\,{\norm{f}}^2}$.