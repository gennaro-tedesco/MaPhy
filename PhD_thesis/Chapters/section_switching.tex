%*****************************************
\section{A reverse picture}
\label{}
%*****************************************

\noindent We have seen that the action of the modular group
with respect to the vacuum state is geometric inside one
interval (Bisognano-Wichmann property, \ref{BiWi}), 
whilst it introduces a mixing among different intervals
described by \cite*{CH:2009}. 
Now we aim to construct a state whose modular group
exactly switches these behaviours, namely whose action
inside one interval is no more geometric, rather it
introduces a mixing among the components described by 
the same matrix as above.

\bigskip
Let us consider, on $\mathcal{A}^N(\I),\,\I\subset\R$, 
the following gauge transformation $\gamma\colon 
\mathcal{A}^N(\I)\to\mathcal{A}^N(\I)$
\[
\gamma(\psi_i(X))=\sum_{r=1}^n {O(X)}_{ir}\,\sqrt{{X_r^{-1}(X)}'}\ 
\psi_r(X)
\]
with the function $X(x)$ as in \eqref{CHfunction} and the 
matrix $O(X)$ exactly given as in \eqref{OonK}. Define the
state $\varphi$ on $\mathcal{A}^N(\I)$ to be $\varphi
\coloneqq \vac\circ{\gamma}^{-1}$; its modular group therefore reads
$\sigma^t_{\varphi}= \gamma\circ\sigma^t_{\vac}\circ{\gamma}^{-1}$
by verification of the \ac{KMS} condition. Explicitly, this gives
\begin{align*}
\sigma^t_{\varphi}(\psi_i(X))&=
(\gamma\circ\sigma^t_{\vac}\circ{\gamma}^{-1})(\psi_i(X))\\
&=(\gamma\circ\sigma^t_{\vac})
\left(\sum_{r=1}^n{O(X)}_{ir}\,{\sqrt{{X_r^{-1}(X)}'}}\right)^{-1}\psi_r(X)\\
\end{align*}
therefore
\begin{multline*}
\sqrt{{X_r^{-1}(X)}'}\,\sigma^t_{\varphi}(\psi_i(X))=
\gamma\left(\sum_{r=1}^n{O(X)}^T_{ir}\sqrt{\delta_t(X)'}\,
\psi_r(\delta_t(X))\right)\\
=\sum_{r,p=1}^n {O(X)}^T_{ir}\,\sqrt{\delta_t(X)'}\,
{O(\delta_t(X))}_{rp}\, \sqrt{{X_p^{-1}(\delta_t(X))}'}\,
\psi_p(\delta_t(X))
\end{multline*}
Since the matrix $O(X)$ satisfies $O(X)^T\,O(\delta_t(X))=
O(t,X)$ we end up with
\[
\sqrt{{X_r^{-1}(X)}'}\,\sigma^t_{\varphi}(\psi_i(X))=
\sum_{p=1}^n O(t,X)_{ip}\,\sqrt{{X_p^{-1}(\delta_t(X))}'}\,
\sqrt{\delta_t(X)'}\,\psi_p(\delta_t(X))
\]
which presents the same mixing appearing in the vacuum modular
flow on the union of $n$ disjoint intervals \cite*[eq (3.1)]{LMR:2009}.
In the same paper the authors also show that a product state
of the form $\varphi_E\coloneqq (\otimes_{k=1}^n \varphi_k)\circ
\chi_E$ has modular group with geometric action within $n$ disjoint
intervals. Therein $\chi_E$ is the isomorphism given by the split
property and $\varphi_k$ are state given by $\varphi_k\coloneqq \vac\circ
\ad U(\gamma_k)$, where $U(\gamma_k)$ implements diffeomorphisms
$\gamma_k\colon z\to z^n$ on $\I_k$ (to be expanded).