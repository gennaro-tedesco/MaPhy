%*****************************************
\subsection{Multi-local diffeomorphisms}
\label{Multi-local diffeomorphisms}
%*****************************************
In the previous paraghraph we showed the construction
of the multi-local current and the resulting 
multi-local fermionisation and gauge transformations.
It is natural to extend the investigation 
to the stress-energy tensor of such a theory and 
look for the corresponding multi-local diffeomorphisms
that are generated. Again, the special case $N=2$ for
symmetric intervals is a guideline, because formulae
simplify and this allows to better understand the 
features and the behaviours without getting lost in 
the nasty coefficient for the general case. Fundamental
is again the characteristic of $\beta$ to preserve the
vacuum state in order to be extended to Wick products:
$\beta\circ\wick{\blank}=\wick{\blank}
\circ\beta$.

\bigskip
We start in the vacuum representation, as usual. 
The real free fermion contains the stress-energy tensor
of central charge $c=1/2$ 
\[
T^{1/2}(z)\coloneqq \frac{-1}{4\pi}\,
\wick{\psi\partial_z \psi}(z)
\]
whereas the complex fermion is, roughly speaking, two
copies thereof, with $c=1$
\[
T^{c=1}(z)\coloneqq \frac{-1}{4\pi}\,
\wick{\psi_1\partial_z \psi_1}(z)+
\frac{-1}{4\pi}\,
\wick{\psi_2\partial_z \psi_2}(z)=
\frac{-1}{4\pi}\,\wick{\phi^*
{\overset{\leftrightarrow}{\partial_z}}\phi}(z).
\]
In terms of the currents, the stress-energy tensor 
can be expressed as
\[
T(z)=\frac{1}{4\pi}\wick{j^2}(z),
\]
that is nothing but the abelian version of \eqref{SET}.

The action of $\beta$ brings to the 
embedded stress-energy tensor which we compute to be
\begin{align*}
\beta\big(T^{1}(z^2)\big)&=\frac{-1}{4\pi}\,
\beta\big(\wick{\phi^*
{\overset{\leftrightarrow}{\partial_z}}\phi}(z)\big)\\[2ex]
&=\frac{-1}{4\pi}\,\wick{\beta(\phi(z^2)^*)
\partial_{z^2}\beta(\phi(z^2))
-\partial_{z^2}(\beta(\phi(z^2)^*))\beta(\phi(z^2))}\\[2ex]
&=\under{\sim\beta(j(z^2))}
{\frac{-1}{4\pi}\,\frac{1}{2z^3}
\wick{(\psi(z)-\psi(-z))(\psi(z)-\psi(-z))}}\\
&+\frac{-1}{4\pi}\cdot\frac{1}{4}\cdot\frac{1}{2z^2}
\wick{\Big((\psi(z)-\psi(-z))\partial_z(\psi(z)+\psi(-z))-\\
&(\partial_z(\psi(z)-\psi(-z)))(\psi(z)+\psi(-z))\Big)}\\[2ex]
&=-\frac{1}{8\pi z^2}\beta(j(z^2))+
\frac{1}{4z^2}\big(T^{1/2}(z)+T^{1/2}(-z)\big),
\end{align*}
expressed as embedding of two real fermions of central 
charge $c=1/2$. The remarkable feature is the presence 
of two delocalised
stress-energy tensors in $z,-z$ plus an additional 
contribution proportional to the embedded 
current. We shall see later that this further term can
be cancelled out by composition with a particular 
automorphism of the current algebra, though.

\bigskip
The real and complex fermions are invariant under 
diffeomorphisms generated by its own stress-energy 
tensor. If $f\colon\s\to\s$ is a general diffeomorphism,
then the unitary operators $V(\gamma_t)=\e^{itT(f)}$ implement
its action as
\begin{equation*}
 \psi'(\gamma(z))=\delta_{\gamma}(\psi(z))=
 V(\gamma)\psi(z)V(\gamma)^*
 =\sqrt{\gamma'(z)}\,\psi(\gamma(z))
\end{equation*}
where $i f(z)/z\in\R$ integrates to
diffeomorphisms given by the one-parameter group 
$\partial_t\gamma_t(z)=-(f\circ\gamma_t)(z)$. 
Also, $\gamma(z)$ is meant as $\gamma_t|_{t=1}(z)$.
Simpler to write down is its 
infinitesimal action $\delta_f^0$ expressed by the 
commutator
\[
\delta_f^0(\psi(z))=i\,\comm*{T(f),\psi(z)}=
\Big(-f(z)\partial_z -\frac{1}{2}f'(z)\Big)\psi(z).
\]
We can now make use of the equation for the 
embedded stress-energy tensor 
\begin{equation}
\label{embedded_SET}
\beta\big(T^1(z^2)\big)=-\frac{1}{8\pi z^2}\beta(j(z^2))+
\frac{1}{4z^2}\big(T^{1/2}(z)+T^{1/2}(-z)\big)
\end{equation}
in order to derive and calculate the 
corresponding multi-local diffeomorphisms.
Similarly to the case of currents we have
the action
$i\,\comm*{\beta\big(T^1(f)\big),\psi(z)}=
(\beta\circ\delta_f^0\circ\beta^{-1})\,\psi(z)$.
The contribution due to the two anti-podal parts
$T^{1/2}(z)$ and $T^{1/2}(-z)$ gives rise to
a term proportional to $\delta_f^0(\psi(z^2))$, 
while the contribution proportional to the 
embedded current gives back two anti-podal terms
in $\psi(z), \psi(-z)$; in details we obtain 
\begin{align*}
i\,\comm*{\beta\big(T^1(f)\big),\psi(z)}&=
(\beta\circ\delta_f^0\circ\beta^{-1})\,\psi(z)\\[2ex]
&=\Big(-\frac{1}{2z}\,f(z^2)\partial_z -\frac{1}{2}
f'(z^2)\Big)\psi(z)\\
&+\frac{1}{4z^2}\,f(z^2)\, 
\big(\psi(z)-\psi(-z)\big).
\end{align*}
Again, we have a mixing of $\psi(z)$ and $\psi(-z)$ 
(due to the current), on top of a geometric action 
due to the stress-energy tensor itself. Of course, in
equation \eqref{embedded_SET} everything is 
expressed in terms of the two real copies 
$\psi_1,\psi_2$ of the free fermion; nevertheless 
one can work it back in terms of the complex
fermion $\phi(z),\phi(z)^*$: in this case, 
when acting with the inverse action $\beta^{-1}$, terms 
proportional to $\wick{\phi^*\partial_z\phi}(z)$
will appear and therefore we will have eventually
mixed pairings $\phi,\phi^*$ expressing some sort
of multi-local ``charged'' conjugation.

As previously mentioned (see chapter  
\ref{The quarks construction}), 
the current algebra possesses
automorphisms of the form 
\[
\rho^q(j(z))=j(z)+\frac{q}{z},\qquad q\in\R
\]
which give rise to charged states $\omega_q
\coloneqq\vac\circ\rho^q$. On the actual Weyl 
operators those automorphisms are realised as
\[
\gamma_q(W(f))=\e^{i\rho^q(j(z))(f)}=
\e^{iq\int_{\s}\dd z \frac{f(z)}{2\pi i z}}\,W(f),
\]
giving rise to different inequivalent representations
whenever one chooses $\gamma_{q_1},\gamma_{q_2}$ with 
$q_1\neq q_2$. Moreover, these automorphisms happen to be
even innerly implemented by unitaries if a real function 
$\varphi$ exists such that $q/z=-i\varphi'(z)$
\[
\gamma_{q}\blank=\ad(W(-\phi))\blank
\]
and $\gamma_q(\alg{\I})=\alg{\I}$. In fact the above equation
can be taken as definition for each representation
$\gamma_q(W(f))$ (\cite{Ca2004}). However, since the  
stress-energy tensor is contained as embedded into
the theory of currents, it turns out that 
composition of $\beta$ with such a $\rho^q$
exactly ``undoes''  the additional contribution
due to the current in \eqref{embedded_SET} if
we shift back $j(z)\mapsto j(z)+q/z$ for 
$q=1/4$. The price to pay is the appearance
of a constant shift $\sim z^{-4}$:
\[
\beta\circ\rho^{1/4}\big(T^1(z^2)\big)=
\frac{1}{4z^2}\big(T^{1/2}(z)+T^{1/2}(-z)\big)+
\frac{1}{64\pi z^4}.
\]
Although we have not introduced the issue yet, we want
to emphasise that the constant term popping up 
is nothing but the Schwarz derivative of the square root 
automatically generated because the stress-energy tensor is a 
quasi-primary field. In fact the above formula will
coincide with the Longo-Xu map applied to a doubled 
theory of stress-energy tensors, as we shall see later on
in paraghraph \ref{Embedding via Longo-Xu map}.

Anyway, it is always very useful to rephrase the 
picture in terms of Fourier modes: as known, 
$T^1(z)=1/2\pi\,\sum_{n\in\Z}L^1_n z^{-n-2}$; then 
the embedding looks like, in terms of Virasoro
generators:
\[
\beta(L_n^1)=-\frac{1}{4}\beta(j_n)+
\frac{1}{2}L_{2n}^{1/2}
\]
for the general case, and 
\begin{multline*}
\beta\circ\rho^{1/4}(L_n^1)=-\frac{1}{4}\beta(j_n)+
\frac{1}{2}L_{2n}^{1/2} +\\ \Big(\frac{1}{4}\beta(j_n)+
\frac{1}{32}\delta_{n,0}\Big)=
\frac{1}{2}L_{2n}^{1/2}+
\frac{1}{32}\delta_{n,0}
\end{multline*}
for the subtracted $q=1/4$ current. Here 
the subtraction of the current modes due to the composition
with a charged automorphism is even more evident. 
In contrast, the first (general)
formula involves also the modes of the current, still, which 
may in turn be expressed in terms of the real Fermi fields.
This emphasises that the embedded diffeomorphisms come
along with embedded gauge transformations, i. e. a mixing
of $\psi(z)$ and $\psi(-z)$, as described before.

\bigskip 
In the Ramond sector the stress-energy tensor presents
an additional term by definition, as shown in equation 
\eqref{SET_vac_vs_R}. Consequently, the action of 
$\beta_{\textrm{R}}$ produces different subtractions 
that cancel the gauge term $\beta(j(z^2))$ which we 
had to deal with in the vacuum sector. 
As such, the formula in the Ramond representation 
becomes easier even without composition with a charged
automorphism of the currents:
\[
\beta_{\textrm{R}}\big(T^1(z^2)\big)=
\frac{1}{4z^2}\Big(\pi_{\textrm{R}}\big(T^{1/2}(z)\big)
+\pi_{\textrm{R}}\big(T^{1/2}(-z)\big)\Big)+
\frac{1}{64 \pi z^4}.
\]

\bigskip 
Finally, a last remark to conclude this section:
the expression of the multi-local 
transformations in terms of Virasoro generator is 
very useful to understand a brand new picture, which
will be pretty convenient when we shall turn to modular
theory. The subgroup $L_0,L_{\pm 1}$ of the Virasoro 
algebra generates the M\"obius group, which in turn
contains rotations, dilations and translations. Its 
multi-local version, given by $\beta(L_0),\beta(L_{\pm 1})$
produces the corresponding multi-local rotations,
dilations and translations. The passage between a single
theory in one interval to a delocalised theory in many
intervals can then by achieved making use of
the same formulae, just taking care of replacing
$L_n\mapsto \beta(L_n)$. The multi-local behaviour
will then be taken into account by the presence of 
$\beta$, automatically, producing mixing among different
components all the time.















%*****************************************
%*****************************************
%*****************************************
%*****************************************
%*****************************************