%*****************************************
\subsection{The symmetric case}
\label{beta symmetric}
%*****************************************
We start with the symmetric case for $N=2$,
namely $z^2\in\I$ and $z,-z\in\I_1,\I_2$ 
respectively. A complex Fermi field $\phi(z^2),
\phi^*(z^2)$ is localised in $\I$ and a real 
Fermi field $\psi(z)$ in $\sqrt{\I}=\I_1\cup 
\I_2$.
\begin{proposition}[\cite{Rehren:2012wa}]
Let $\phi$ and $\psi$ stand for the complex and real 
fermion in their vacuum representation, as stated 
above. The linear map 
\begin{equation}
\label{beta}
\beta\colon\mathcal{A}^2(\I)
\to\alg{\sqrt{I}}
\end{equation}
given by 
\begin{align}
\phi(z^2)&\mapsto\frac{1}{2}
\left(\psi(z)+\psi(-z)\right)
\label{beta_complex}\\[1ex]
\phi^*(z^2)&\mapsto\frac{1}{2z}
\left(\psi(z)-\psi(-z)\right)
\end{align}
for $z\in\s$, induces an isomorphism of $\textrm{CAR}$
algebras preserving the vacuum state on the different 
algebras: $\vac^{(1)}\circ\beta=\vac^{(2)}$. Note that 
the map is well defined because the right hand sides
are invariant under $z\mapsto -z$.
\end{proposition}
\begin{proof}
We start showing the inverse of such a map: clearly, 
summing up the two sides of the equations we obtain
\begin{align*}
2\,\psi(z)&=2\,\beta(\phi(z^2)) + 2z\,\beta(\phi^*(z^2))\\
2\,\psi(-z)&=2\,\beta(\phi(z^2)) - 2z\,\beta(\phi^*(z^2))\\
\end{align*}
therefore the inverse relation reads
\[
\beta^{-1}\left(\psi(\pm z)\right)=\phi(z^2)\pm z\phi^*(z^2).
\]
The adjoint relation is immediate:
\[
\beta(\phi(z^2))^*=z^2\,\beta(\phi^*(z^2)),
\]
hence $\beta$ preserves the adjoints too.
In terms of the two copies 
$\phi(x)=\left(\psi_1(x)+i\psi_2(x)\right)/\sqrt{2}$
the map $\beta$ can be written as
\begin{align*}
\psi_1(z^2)&\mapsto\frac{1}{2\sqrt{2}}\,\psi(z)
\left(1+\frac{1}{z}\right)+\frac{1}{2\sqrt{2}}\,\psi(-z)
\left(1-\frac{1}{z}\right)\\[1ex]
\psi_2(z^2)&\mapsto\frac{1}{2\sqrt{2}}\,\psi(z)
\left(1-\frac{1}{z}\right)+\frac{1}{2\sqrt{2}}\,\psi(-z)
\left(1+\frac{1}{z}\right)\\
\end{align*}
with inverse given by
\[
\beta^{-1}\left(\psi(\pm z)\right)=\
\psi_1(z^2)(1\pm z) +i\psi_2(z^2)(1\mp z).
\]
Now we turn to the vacuum preserving features. The 
simplest proof proceeds by brute force plugging the 
right hand side of \eqref{beta} into the two-point 
function and evaluating the result:
\begin{align*}
\vac\circ\beta\left(\phi^*(z^2)\phi(w^2)\right)&=
\vac\left(\beta(\phi^*(z^2))\beta(\phi(w^2))\right)\\
&=\vac\left(\frac{1}{2z}(\psi(z)-\psi(-z))\right. \cdot
\left.\frac{1}{2}(\psi(w)+\psi(-w))\right)
\end{align*}
this brings four contributions:
\begin{multline*}
\frac{1}{4z}\left(\vac(\psi(z)\psi(w))-\vac(\psi(-z)\psi(w))\right. \\
+\left. \vac(\psi(z)\psi(-w))-\vac(\psi(-z)\psi(-w))\right)
\end{multline*}
which can be summed up using the formula for the vacuum 
two-point function for the real chiral Fermi field 
\begin{align*}
\vac\circ\beta\left(\phi^*(z^2)\phi(w^2)\right)&=
\frac{1}{4z}\left( \frac{1}{z-w}-\frac{1}{-z-w}\right. 
+\left. \frac{1}{z+w}-\frac{1}{-z+w}\right)\\
&=\frac{1}{4z}\,2\,\left(\frac{1}{z-w}+\frac{1}{z+w}\right)\\
&=\frac{1}{2z}\cdot\frac{2z}{z^2-w^2}=\frac{1}{z^2-w^2}\\
&=\vac\left(\phi^*(z^2)\phi(w^2)\right)
\end{align*}
as to be proven. By exploiting Wick theorem, this 
equality extends to all $n$-points functions, since 
these are just sums of products of two-point functions,
see equation \eqref{wick}. As already pointed out, the 
standard anti-commutation relations follow from this
correlation function, therefore they remain preserved as
well.

\bigskip 
Another proof proceeds by simply looking at Fourier modes. 
In the vacuum representation, where we assume the fields are 
evaluated, we have 
\[
 \phi(z^2)=\sum_{r\in\Z+1/2}\phi_r\,
 {(z^2)}^{-r-1/2},\quad
 \psi(z)=\sum_{r\in\Z+1/2}\psi_r\,
 z^{-r-1/2}.
\]
A simple look at the right hand sides of \eqref{beta} 
displays that, for example,
\begin{align*}
\beta\left(\phi(z^2)\right)&=\sum_{r\in\Z+1/2}\beta(\phi_r)\,
 {(z^2)}^{-r-1/2}=
\frac{1}{2}\left(\psi(z)+\psi(-z)\right)\\[1ex]
&=\frac{1}{2}\sum_{r\in\Z+1/2}\phi_r\,z^{-r-1/2}
\left(1+(-1)^{-r-1/2}\right) \\[1ex]
&=\sum_{p\in\Z}\psi_{2p-1/2}{(z^2)}^{-p}=
\sum_{r\in\Z+1/2}\psi_{2r+1/2}{(z^2)}^{-r-1/2}
\end{align*}
therefore the isomorphism appears as a relabelling 
of the Fourier modes $\phi_r\mapsto\psi_{2r+1/2}$.
Similarly for the adjoint field we have 
$\phi^*_r\mapsto\psi_{2r-1/2}$; the variable $r$
runs into $\Z+1/2$ for the vacuum representation.
In terms of these Fourier modes the anti-commutation
relations read
\[
\ant{\phi_r,\phi^*_s}=\ant{\psi_r,\psi_s}=\delta_{r+s,0}
\]
and a simple look shows that
\[
\ant*{\beta(\phi_r),\beta(\phi^*_s)}=
\ant{\psi_{2r+1/2},\psi_{2s-1/2}}=\delta_{2r+1/2+2s-1/2,0}
=\delta_{r+s,0};
\]
the vacuum state is the only state which is annihilated 
by all $\psi_r, r>0$ and by all $\phi_r,\phi_r^*, r>0$,
respectively. Of course the relabelling does not 
change these conditions and ergo the vacuum is still sent 
into itself. The adjoint relations in terms of modes are
$\phi_r^*=(\phi_{-r})^*$ and we can easily see that, by 
making use of $\beta(\phi(z^2))^*=z^2\,\beta(\phi^*(z^2))$
multiplication by $z^2$ becomes a shift by $-1$ in terms 
of modes, and this is exactly $2r+1/2-1=2r-1/2$, indeed 
what happens to $\phi^*_r$ under the action of $\beta$.

\bigskip 
We have thus shown that $\beta$ is an isomorphism that 
preserves the vacuum state, both in the ``local'' setting
(by looking at the two-point function) and from the 
algebraic perspective of anti-commutators.
This is due to change of localisation from the point 
$z^2$ to the points $z,-z$ with suitable coefficients that
must adjust the form of two-point function eventually. We shall 
see later on that such coefficients will acquire a more
general form described by the Casini-Huerta function 
\eqref{CHfunction} in the context of modular theory and this 
will come as a very special feature; with any other function 
$\I_k\to \I$ the statement is no more true. The 
reduction to special symmetric case brings back the form 
we have just analysed. However, it is 
interesting to show what the map $\beta$ becomes on the 
real line, instead. Since $\R=C^{-1}(\s\setminus\set{-1})$ 
then 
\[
\beta^{\R}=C^{-1}\circ\beta^{\s}\circ C.
\]
The square root assignment $z^2\mapsto\sqrt{z^2}=\pm z$
becomes, after Cayley transform, 
$C^{-1}(z^2)=q(x)=2x/(1-x^2)$, whose two 
``square roots'' are the points $x=C^{-1}(z)$ and
$-1/x=C^{-1}(-z)$.
We have that, on the real line, $\beta^{\R}$ is
\begin{align*}
\phi(q(x))&\mapsto\frac{1}{q(x)}\cdot\frac{1}{1-ix}
\Big(x\psi(x)+i\psi(-1/x)\Big)\\
\phi^*(q(x))&\mapsto\frac{1}{q(x)}\cdot\frac{1}{1+ix}
\Big(x\psi(x)-i\psi(-1/x)\Big)
\end{align*}
and the inverse relation is simply
\[
\beta^{-1}(\psi(x))=\frac{1-ix}{1-x^2}\,\phi(q(x)) +
\frac{1+ix}{1-x^2}\,\phi^*(q(x)).
\]
One might still ask to show why the vacuum correlation 
function is preserved on the real line too; this can be
easily proven by brute force, though we prefer to show a
more elegant solution as follows: the vacuum states on the 
real line and on the circle are related via $\vac^{\R}=
\vac^{\s}\circ C$:
\begin{align*}
\vac^{\R}\circ\beta^{\R}&=\vac^{\s}\circ C\circ\beta^{\R}\\
&=\vac^{\s}\circ C\circ C^{-1}\circ\beta^{\s}\circ C\\
&=\vac^{\s}\circ\beta^{\s}\circ C\\
&=\vac^{\s}\circ C=\vac^{\R}
\end{align*}
as to be proven. \qedhere
\end{proof}
With a little care the result we have just presented can be 
straightforwardly generalised to the case of symmetric $N$-intervals. 
Before we do so, we notice that
a closer look to $\beta$ brings the following inspection to 
the coefficients appearing on the right hand sides:
\begin{align*}
\phi(z^2)&\mapsto \frac{z^0}{2}
\left(\psi(\omega^0 z)+\psi(\omega^1 z)\right)\\
\phi^*(z^2)&\mapsto \frac{z^{-1}}{2}
\left(\psi(\omega^0 z)-\psi(\omega^1 z)\right)
\end{align*}
where we have introduced the roots of the unity $\omega^{0,1}=\pm 1$
as in the fashion previously described. Even more compact
is the form:
\begin{equation*}
\phi^{(k)}(z^2)\mapsto\frac{z^{1-k}}{N}\,\sum_{j=0}^{N-1}
\omega^{(1-k)j}\,\psi(\omega^j z)\qquad k=1,2.
\end{equation*}
Let us now take any symmetric $N$-interval with arbitrary $N$, 
exploiting the form of $\sqrt[N]{z^N}$ with $N$
solutions $z_1,\ldots,z_N$ such that $z_k=\omega^k z\in\I_k$.
Similarly we choose $N$ real Fermi fields lying in $\alg{\I}$ 
which we can pairwise combine into $N/2$ complex Fermi fields:
in formulae we assign $\psi^{(1)}(z^N),\ldots,\psi^{(N)}(z^N)$
such that ${{\phi^{(k)}}^*(z^N)}=\phi^{(N+1-k)}(z^N)$.
Then we just have to propose the same ansatz 
for a symmetric $N$-interval $\beta\colon
\mathcal{A}^N(\I)\to\alg{\sqrt[N]{\I}}$
\begin{equation}
\label{beta_any}
\phi^{(k)}(z^2)\mapsto\frac{z^{1-k}}{N}\,\sum_{j=0}^{N-1}
\omega^{(1-k)j}\,\psi(\omega^j z)\qquad k=1,\ldots,N.
\end{equation}
In terms of the initial fields and in terms of Fourier 
modes this becomes 
\[
\psi^{(k)}(z^N)\mapsto\sum_{r\in\Z+1/2}\psi_{1/2-k + (r+1/2)N}\,
{(z^N)}^{-r-1/2}
\]
which corresponds in turn to a relabelling 
of generators $\psi^{(k)}_r\mapsto\psi_{1/2-k + (r+1/2)N}$.
That this is still a vacuum preserving isomorphism can be
again verified by noting that the renumbering of generators 
does not affect the vacuum, or by direct computation of the 
two-point functions, along the same lines as before.







%*****************************************
%*****************************************
%*****************************************
%*****************************************
%*****************************************
