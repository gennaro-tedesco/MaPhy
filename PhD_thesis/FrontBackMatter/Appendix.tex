%*****************************************
\chapter{Appendix}
\label{ch:Appendix}
\minitoc\mtcskip
%*****************************************

\section{On the passage to two-dimensional models}
As aforementioned, we focused our attention on 
one-dimensional chiral models where fields depend
on the light-cone variables $x_{\pm}$ only. 
Taking two such theories, respectively described by the nets
of algebras $\alg{I_+}, \alg{I_-}$ (with obvious
understanding of notations), the chiral two-dimensional
model is given by the tensor product $\alg{\mathcal{O}}=\alg{I_+}\otimes
\alg{I_-}\subset\mathcal{B}(\mathcal{O})$, with the space-time region 
$\mathcal{O}$ given by $I_+\times I_-$. For the tensor product
theory of observables the vacuum state (actually its
\ac{GNS} representation) is the tensor product 
$\Omega\otimes\Omega$ (acting on $\hil\otimes\hil$)
and therefore the modular theory
derived thereof decomposes into tensor products as 
well. In fact the anti-linear operator \eqref{Tomita}
becomes $S_0\colon a(\Omega\otimes\Omega)\mapsto 
a^*(\Omega\otimes\Omega),\,a\,\in\alg{O}$ and the 
corresponding modular operator is the tensor product
$\Delta_{\mathcal{O}}^{it}=\Delta_+^{it}\otimes
\Delta_-^{it}$, giving rise to modular automorphisms group 
as $\sigma^t=\sigma^t_{I_+}\otimes\sigma^t_{I_-}$
by verification of the \ac{KMS} condition.
In particular $\mathcal{O}$ are double cones
if $\I_+,\I_-$ are two intervals, the forward light 
cone $\mathcal{V}_+$ as $\R_+\times\R_+$ and the
right wedge as $\R_+\times\R_-$. Replacing everywhere
$\R_+\to\R_-$ gives the backward light cone and the 
left wedge, respectively.


\section{More on the geometric action of modular groups
for special regions}
It has been pointed out that the most of the modular theory
relies on the result of Bisognano and Wichmann 
\eqref{BiWi} expressing the modular group and the 
modular conjugation for Wightmann fields localised in 
wedge regions. This results allows some sort of
generalisations to the cases of space-time regions
that can be obtained as geometric transformations of
the wedges, provided the vacuum vector to be invariant
under such transformations. In particular, we shall recall
here a remarkable result found out by \cite{HisLon:1982}
for double cones and massless scalar fields obeying the 
Klein-Gordon equation. 

The original result by Hislop and Longo refers to the 
four-dimensional case, but nevertheless it can be transferred to 
two dimensions. In particular, in order to do so, the scalar
field $\phi(f)$ has to be smeared with test functions 
which are light-cone variables derivatives, namely 
$f=\partial_{\pm} g$, $g$ being an appropriate test 
function. We proceed by noticing that 
double cones $\mathcal{O}$ can be mapped 
into wedge regions $\wed$ by means of the inversion map
\[
\rho\colon(x^0,x^1)\mapsto\rho(x^0,x^1)=
\frac{1}{\abs{x}^2}\big(-x^1,-x^0\big),\quad 
\abs{x}^2=(x^0)^2-(x^1)^2.
\]
Given $\phi(f)$ as a solution of the Klein-Gordon equation
$\phi(\Box f)=0$, with $f$ a function of the said form, 
the authors showed that this action can be implemented
on the one-particle Hilbert space $\hil$ through $U_{\rho}$
(\cite{HisLon:1982})
\[
U_{\rho}\phi(f)\Omega=\phi(f_{\rho})\Omega
\]
where $f_{\rho}(x)=-{(\abs{x}^2)}^{-3}f(\rho(x))$. Due
to the conformal symmetry, $U_{\rho}$ extends to a unitary
operator $\Gamma(U_{\rho})$ onto the Fock space $\mathcal{F}(\hil)$ 
preserving the vacuum state whose action is given by
$\Gamma(U_{\rho})\phi(f)\Gamma(U_{\rho})^*=
\phi(f_{\rho})$ and gives rise to 
an isometry between the Weyl algebra of the double cone
and the wedge region $\Gamma(U_{\rho})\alg{\mathcal{O}}
\Gamma(U_{\rho})^*=\alg{\rho(\mathcal{O})}=\alg{\wed}$.
Since such a unitary preserves the vacuum state, it does
also connect the modular objects corresponding to
$\alg{\mathcal{O}}$ and $\alg{\wed}$ as
\begin{align*}
J_{\mathcal{O}}&=\Gamma(U_{\rho})\,J_{\wed}\,\Gamma(U_{\rho})^*\\
\Delta^{it}_{\mathcal{O}}&=\Gamma(U_{\rho})\,\Delta^{it}_{\wed}\,
\Gamma(U_{\rho})^*
\end{align*}
resulting in a geometric action of the modular group within
the double cone given in terms of conformal transformations as
\[
x_{\pm}=\frac{1+x_{\pm}-\e^{-s}(1-x_{\pm})}
{1+x_{\pm}+\e^{-s}(1-x_{\pm})}
\]
with $x_{\pm}$ the standard light-cone variables and 
$s$ a real dilation parameter. 

A similar geometric transformation can be introduced to 
map the open double cone $\mathcal{O}'$ into the forward light cone
$\mathcal{V}_+$ so that the algebras $\alg{\mathcal{O}'}$ and 
$\alg{\mathcal{V}_+}$ are equivalent by means of the 
unitary $\Gamma(T(1/2)U_{\rho}T(-1))$, where $T(\lambda)$
implements time translations. The relations
\begin{align*}
J_{\mathcal{O}'}&=\Gamma\big(T(1/2)U_{\rho}T(-1)\big)
\,J_{\mathcal{V}_+}\,\Gamma\big(T(1/2)U_{\rho}T(-1)\big)^*\\
\Delta^{it}_{\mathcal{O}'}&=\Gamma\big(T(1/2)U_{\rho}T(-1)\big)
\,\Delta^{it}_{\mathcal{V}_+}\,
\Gamma\big(T(1/2)U_{\rho}T(-1)\big)^*
\end{align*}
reproducing a well known result of Buchholz \cite{Buchholz:1977} 
stating that the modular operator and conjugation for the 
forward light cone are respectively given by 
dilations and CPT inversion mapping $\mathcal{V}_+$ onto
the backward light cone $\mathcal{V}_-$, that is 
$\Delta_{\mathcal{V}_+}^{it}=\Gamma(\delta(2\pi \lambda))$ and
$J_{\mathcal{V}_+}=\Gamma(-CPT)$.

\bigskip
In case of massive theories the action of the modular group 
is known only for wedge regions, again due to the 
Bisognano-Wichmann property. Since massive theories are in
general not conformally invariant this result cannot be transferred
to double cones and similar regions, unlike the massless cases.
It can be shown that the general action has to be non-local and
presumably given in terms of pseudo-differential operators; 
in particular if $\delta=\partial_t \Delta^{it}|_{t=0}$ is the 
infinitesimal generator of the modular group, then 
$\delta=\delta_{0} + \delta_m$, where $\delta_0$ is the standard
massless generator and $\delta_m$ is expressed in terms of the
action of a pseudo-differential operator depending on the mass.
We refer the reader to \cite{Saf:2006, Yngv} for progresses
in these directions.



\section{Correlations functions in conformal field theory}
\label{correlations functions}
\noindent In this section we are going to have a closer look at
the explicit form of correlations functions in
conformal field theory. In particular we shall see
that conformal invariance, especially in low dimensions,
poses strong restrictions to the form of such 
correlations functions and almost fixes them all, 
up to some constants, in the case of two and three
points functions.

In two dimensions the conformal group reduces to the set
of holomorphic and anti-holomorphic functions and in
the special case of chiral theories we are allowed
to look at each copy singularly. This means that, as 
already pointed out, the conformal group decomposes
into two copies, each one of them is generated by the
Virasoro algebra \eqref{Vir}
\[
\comm{L_n,L_m}=(n-m)L_{n+m} + \frac{c}{12} m(m^2-1)\,
\delta_{n+m}\bm{1}.
\]
A special class of fields is given by primary fields,
as described in \ref{Primary fields}, which have the special
property to transform as in \eqref{finite primary}
\[
\phi(z)={\left(\frac{dg}{dz}\right)}^h\phi'(g(z))
\]
under conformal transformations, $h$ being the conformal
dimension of the fields. Here $z\mapsto g(z)$ is the 
conformal mapping for either of the holomorphic or
anti-holomorphic variables, one at a time. This corresponds
to the finite exponential form of the infinitesimal 
commutation relations between such fields and the generators
of the Virasoro algebra 
\[
\comm{L_n,\phi(z)}=h(n+1)z^n\phi(z) + z^{n+1}\partial_z \phi(z).
\]
Correlations functions are defined as vacuum expectation values
of products of fields (in the sense of tempered distribution)
\[
 w^n(x_1,\ldots,x_n)\coloneqq
 \scal{\Omega,\phi_1(x_1)\ldots\phi_n(x_n)\Omega}
\]
and the idea is now that, if we require the above quantities to
be invariant under conformal transformations, we may fix
the form of such functions up to some degrees of freedom.
In particular we have to impose that $w^n(x_1,\ldots,x_n)=
w'^n(x'_1,\ldots,x'_n)$, namely
\[
\scal{\Omega,\phi_1(x_1)\ldots\phi_n(x_n)\Omega}=
\scal{\Omega,\phi'_1(x'_1)\ldots\phi'_n(x'_n)\Omega}
\]
where, with obvious understanding of notations,
$\phi'_k(x'_k)$ is the new field after a change under
conformal transformations given by 
$\phi'(x')=U\,\phi(x)\,U^*$.
Exploiting such formula we find interesting results.

\subsection{The two-point function}
\label{the two-point function}
Let us concentrate first on the two-point function 
for primary fields
$w^{(2)}(x_1,x_2)=\scal{\Omega,\phi_1(x_1)
\phi_2(x_2)\Omega}$ and let
us impose invariance under translations, dilations
and special conformal transformations keeping in 
mind that primary fields change as
\begin{align*}
i\comm{P,\phi(x)}&=\partial_x\phi(x)\\
i\comm{D,\phi(x)}&=(x\partial_x+h)\phi(x)\\
i\comm{K,\phi(x)}&=(x^2\partial_x+2hx)\phi(x)
\end{align*}
where we have seen in chapter \ref{The Moebius group} that $P,D,K$ can
be expressed in terms of $L_0, L_{\pm 1}$.
Invariance under translations requires
\[
\scal{\Omega,\comm{P,\phi_1(x_1)\phi_2(x_2)}\Omega}=0;
\]
using $\comm{A,BC}=B\comm{A,C}+\comm{A,B}C$ we are led to
\begin{align*}
0&=\scal{\Omega,\phi_1(x_1)\,\comm{P,\phi_2(x_2)}\Omega}+
\scal{\Omega,\comm{P,\phi_1(x_1)}\,\phi_2(x_2)\Omega}\\
&=(\partial_{x_1}+\partial_{x_2})w^{(2)}(x_1,x_2)\\
\end{align*}
and thus $w^{(2)}(x_1,x_2)$ depends only on the difference
of the two variables $w^{(2)}(x_1,x_2)=w^{(2)}(x_1-x_2)$, 
which is the standard form required by translations invariance.
Along the same lines, for dilations invariance we have
\begin{align*}
0&=\scal{\Omega,\comm{D,\phi_1(x_1)\phi_2(x_2)}\Omega}\\
&=(x_1\partial_{x_1}+h_1+x_2\partial_{x_2}+h_2)w^{(2)}(x_1,x_2)\\
\end{align*}
introducing the variable $x=x_1-x_2$, as we have seen 
above, we obtain $(x\partial_x + h_1+h_2)w^{(2)}(x)=0$ 
and thus
\[
\frac{1}{w^{(2)}(x)}\,\dd w^{(2)}(x)=-(h_1+h_2)\,\frac{1}{x}\,\dd x
\]
which integrates to 
${w^{(2)}(x)}=c_{12}\,x^{-(h_1+h_2)}$;
keep in mind that we want the correlations functions to
diverge whenever the two points coincide, therefore for
$x\to 0$. This implies that $h_1+h_2$ must be positive.
Last, but not the least, we impose invariance under special
conformal transformations
\begin{align*}
0&=\scal{\Omega,\comm{K,\phi_1(x_1)\phi_2(x_2)}\Omega}\\
&=(x_1^2\partial_{x_1}+2h_1 x_1+x_2^2\partial_{x_2}+2h_2 x_2)
w^{(2)}(x_1,x_2).\\
\end{align*}
To help the computation we can plug in the form
${w^{(2)}(x)}=c_{12}\,x^{-(h_1+h_2)}$
and work it out:
\begin{align*}
0&=\left(x_1^2 (-1)(h_1+h_2)(x)^{-1} + 2h_1 x_1+\right.
\left. x_2^2 (h_1+h_2)(x)^{-1} + 2h_2 x_2\right){w^{(2)}(x)}\\
&=\left((x)^{-1}(h_1+h_2)(x_2^2-x_1^2) + 2h_1 x_1 + 2h_2 x_2\right)
{w^{(2)}(x)}\\
&=\left(-(x_1+x_2)(h_1+h_2) + 2h_1 x_1 + 2h_2 x_2\right){w^{(2)}(x)}\\
&=(h_1-h_2)(x_1-x_2){w^{(2)}(x)}
\end{align*}
interestingly enough then, the two fields are correlated only
if the two scaling dimensions coincide, $h_1=h_2$. Of course,
all the calculations must be intended in the sense of distribution,
therefore for primary fields the two-point function takes the form 
\[
w^{(2)}(x_1-x_2)=\lim_{\varepsilon\to 0^+} 
\left(\frac{c_{12}}{x_1-x_2-i\varepsilon}\right)^{2h}
\]
where the normalisation constant $c_{12}$ is the only parameter left free and
can be calculated by imposing further requirements, as well
as positivity of the scalar product in the Hilbert space
(in the sense of operators) and spectrum conditions. It is
straightforward now to derive back the expression of the
two-point function for fermions and currents: substitution
of $h=1/2$ and $h=1$ gives the results we have already 
stated by performing explicit calculations on the fields 
themselves.

\subsection{The three-point function}
Similar arguments can be undertaken for the three-point
function too. Again, translations invariance states
\[
\scal{\Omega,\comm{P,\phi_1(x_1)\phi_2(x_2)\phi_3(x_3)}\Omega}
=(\partial_{x_1}+\partial_{x_2}+\partial_{x_3})w^{(3)}(x_1,x_2,x_3)=0
\]
meaning that $w^{(3)}(x_1,x_2,x_3)$ must depend on the pairwise
difference of the variables $w^{(3)}(x_1-x_2,x_1-x_3,x_2-x_3)$.
Dilations invariance brings homogeneity
\[
w^{(3)}(x_1,x_2,x_3)=\frac{c_{123}}{(x_1-x_2)^a\cdot(x_1-x_3)^b
\cdot(x_2-x_3)^c}
\]
and special conformal invariance fixes the exponents $a,b,c$
to be $a=h_1+h_2-h_3$, $b=h_2+h_3-h_1$, $c=h_3+h_1-h_2$; thus
fields of different scaling dimension may still have 
non-vanishing three-point function.

\bigskip
Higher correlations functions might in principle be similarly
derived, with the only difference that in this case 
M\"obius invariance does not 
give enough restrictions as in the case of two and three points
functions. Nevertheless the idea is always to start with
the $n$-point function $w^{(n)}(x_1,\ldots,x_n)$ and impose invariance under 
a general change after conformal transformations; for each
of the M\"obius generators we have, in principle:
\[
\scal{\Omega,\comm{L_n,\phi_1(x_1)\ldots\phi_n(x_n)}\Omega}=0
\]
and by multiple application of the Leibniz rule for
commutators the above equation can be turned into
a differential equation for the correlator as
$\textrm{D}\,w^{(n)}(x_1,\ldots,x_n)=0$, where 
$\textrm{D}$ is a differential operator, depending
on case by case. Such differential equations are
usually referred to as ``Ward identities'' and
can be used to test concrete models, although, as
we said, they do not restrict enough the form
of the Wightman $n$-points functions.




%*****************************************
%*****************************************
%*****************************************
%*****************************************
%*****************************************
