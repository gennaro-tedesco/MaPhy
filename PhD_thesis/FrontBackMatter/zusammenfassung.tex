%*******************************************************
% Abstract
%*******************************************************
%\renewcommand{\abstractname}{Abstract}
\pdfbookmark[1]{Abstract}{Abstract}
\begingroup
\let\clearpage\relax
\let\cleardoublepage\relax
\let\cleardoublepage\relax

\chapter*{Abstract}
The following thesis deals with the modular theory of 
Fermi fields in low dimensions; in particular, making 
use of the algebraic approach to quantum field theory,
we have investigated the behaviour of two-dimensional 
theories which split into two separate copies of chiral
fields, each one of them depending on one lightray variable 
at a time only.

The remarkable result we have found is the existence of 
a vacuum preserving isomorphism $\beta$ connecting the 
vacuum states between the algebra of $N$ Fermi fields 
localised in one single interval $\I$ and the algebra 
of one Fermi field localised in $N$ disjoint intervals 
$\textrm{E}_N=\I_1\cup\ldots\cup \I_N$. Since this 
map preserves the vacuum states, it therefore intertwines 
the respective modular groups; as a result, the modular 
automorphism flow for a Fermi field localised in several 
intervals turns out to mix the field among different 
points, with the mixing itself being described through 
suitable differential equations. Moreover, using the fact that
Wick products are as well preserved, one can even embed 
via $\beta$ the sub-theories of local observables,
as currents and the stress-energy tensor. Consequently,
since the isomorphism $\beta$ is multi-local, a 
new class of multi-local gauge transformations and 
diffeomorphisms arise. 

Interestingly enough, such characterisation of the 
modular group for multi-local algebras was already 
presented by \cite{CH:2009} using different
techniques, and so far it is a special feature of 
free Fermi fields only (although outlooks of generality
are fascinating to investigate).

\bigskip
The isomorphism that we have found is deeply related to 
the split property and the way fields transform under 
diffeomorphism covariance. In particular, it only differs 
from the action of diffeomorphisms by a gauge transformation,
whose features we have characterised in the cases at hand,
namely for the local algebras of Fermi fields, currents and 
stress-energy tensor.


\vfill

\pdfbookmark[1]{Zusammenfassung}{Zusammenfassung}
\chapter*{Zusammenfassung}
Die folgende Doktorarbeit befasst sich 
mit der Modulartheorie von Fermifeldern in niedrigen Dimensionen;
insbesondere untersuchen wir das Verhalten der chiralen Felder,
nachdem Felder in zwei Dimensionen 
in zwei ein-dimensionale Lichtstrahlkomponenten 
zerlegt worden sind. Wir wenden den algebraischen
Zugang zur Quantenfeldtheorie an, in dem man sich mit 
lokalen Algebren befasst. 

Wir finden einen Isomorphismus $\beta$ zwischen 
der Algebra von $N$ Fermifeldern, die in einem einzelnen Interval
$\I$ lokalisiert sind, und der Algebra eines Fermifelds, 
das in mehreren verschieden Intervallen $\textrm{E}_N
=\I_1\cup\ldots\cup \I_N$ lokalisiert ist, der den Grundzustand
erh\"alt. Daher verkn\"upft dieser die 
korrespondierenden Grundzustandmodulargruppen. Weil 
dieser Isomorphismus nicht-lokal ist, ergibt sich eine Mischung 
f\"ur die Modulargruppe der Multi-Interval-Algebra, 
die das Feld in verschiedenen Punkten in den unterschiedlichen 
Intervallen mischt.

\bigskip
Diese Characterisierung der Modulargruppen f\"ur die 
Multi-Interval-Algebra ist nur f\"ur freie chirale Fermifelder bekannt. 
Da dieser Isomorphismus auch Wick Produkte erh\"alt, k\"onnen auch
lokale Observablen, wie die Str\"ome und der Energie-Impuls-Tensor, 
damit eingebettet werden. Wegen dieses Merkmals kann man 
multi-lokale Eichsymmetrien und Diffeomorphismen generieren,
deren Verhalten wir auch untersucht haben.

\bigskip 
Der Isomorphismus, den wir gefunden haben, setzt sich interessanterweise
zusammen aus dem Split-Isomorphismus einer geeigneten Wirkung
der Diffeomorphismen und einer Eichtransformation. 
Das gleiche Verhalten kann man auch auf die 
Untertheorien der Str\"omen und des Energie-Impuls-Tensors 
einschr\"anken, was wir uns im letzten Kapitel angesehen haben.


\endgroup			

\vfill