%*****************************************
\addcontentsline{toc}{chapter}{Conclusions and outlooks}
\chapter*{Conclusions and outlooks}
\label{conclusions}
%*****************************************
The ideas that we have shown allow many more future 
perspectives, both from the point of view of modular 
theory itself and for what concerns the investigation
of embedded theory of observables as currents models
and so forth. The first bunch of questions arising 
are related to possible generalisations of the result 
of Casini and Huerta to the free Bose fields, trying to
look at the corresponding relation between density matrix
(containing the modular Hamiltonian) and correlators, that
in principle should give back the modular ``time evolution'' 
for bosons localised in 
disjoint intervals, as similar to
the case of Fermi fields in two dimensions.

Then one could try to extend such results to the massive
case, again both for the Bose fields and for the Fermi ones,
taking advantage of some already existing results 
(mainly by \cite{FigGui:1989, Saf:2006}) who showed that
in the massive cases the action of the modular group 
for the free Bose field in particular space-time regions
is given in terms of a pseudo-differential operator 
depending on the mass. Here the techniques mostly go in the
direction of functional analysis and differential equations,
although the insights from the algebraic approach can still
be instrumental.

\bigskip 
The understanding of the free Bose fields automatically
leads to the characterisation of the currents models and
their modular theory, also moving the interest to the 
study of loop group models and their representations. 
In particular, we have seen that suitably chosen gauge 
transformations help us to trace modular theory back to
an underlying isomorphism between algebras and perhaps
a complete understanding of such gauge maps in a more
general setting (maybe of non-standard form) 
can help very much to have further developments in
that area. Interestingly enough, the conjecture that the 
existence of a vacuum preserving isomorphism is connected to
the absence of sectors is still an open problem.

\bigskip 
Very challenging is also the characterisation of modular 
theory in higher dimensions. The two-dimensional case can be
pretty much derived taking the tensor product of two one-dimensional
theories and therefore all the modular objects can be easily
derived. On the other hand, models in three and four dimensions
are a wide open area of investigation and the first attempt
could be trying to understand which features remain the same and
which other features present totally different behaviours instead.



%*****************************************
%*****************************************
%*****************************************
%*****************************************
%*****************************************
