%*******************************************************
% Abstract
%*******************************************************
%\renewcommand{\abstractname}{Abstract}
\pdfbookmark[1]{Introduction}{Introduction}
\begingroup
\let\clearpage\relax
\let\cleardoublepage\relax
\let\cleardoublepage\relax

 \addcontentsline{toc}{section}{Introduction}
 \chapter*{Introduction}
 The main theoretical ingredient of algebraic quantum field
 theory is the concept of field, which is supposed to
 implement the principle of locality. Observables, identified
 with the quantities that can be experimentally measured in a 
 laboratory, must satisfy Einstein causality and additional
 physical requirements that are seen to be realised in 
 nature. Fields therefore appear as the building blocks
 in order to construct such observables and, though they
 may themselves be observables, they need not to. The idea 
 lying at the basis of quantum field theory is the 
 assignment of fields to each space-time region, where
 events are supposed to take place. This reflects into 
 the assignment of a net of algebras onto the Minkowski
 space; physical measurements correspond, roughly speaking,
 to states on the algebras and all the most important physical
 quantities experimentalists are interested in can usually
 be traced back to the evaluations of scalar product or 
 particular combinations thereof, as for example correlations
 functions and scattering amplitudes: notable in this 
 sense is the Lehmann-Symanzik-Zimmermann formula reducing 
 scattering amplitudes to time-ordered correlations functions
 and their poles.
 
 The algebraic approach to quantum field theory deals with
 the mathematical properties of all these ingredients 
 from the point of view of operator algebras. 
 A marvelous walkthrough these aspects is provided by 
 \cite{Haag} and \cite{roberts:2004} who give complete
 explanations of why this is a necessary issue. The developement
 of such a formalism is the key tool to the understanding of
 quantum field theory itself and encodes almost all the 
 features that we find as realised in nature. Many results
 have been achieved thanks to the possibility to handle 
 these mathematical tools, especially after very important
 insights by Takesaki and Tomita, \cite{Borchers:1999},
 \cite{Tak:1970}, \cite{TAKII:2002}, who reduced the origin
 of space-time symmetries to abstract properties of von 
 Neumann algebras, opening a brand new research field
 consequently.
 
 \bigskip 
 A very important role in physics is played by systems 
 which exhibit special symmetries, because this characteristic
 helps a lot to reduce their complexity. In particular 
 we have been concerned with models being symmetric under
 conformal transformations, that is the set of transformations 
 preserving the angles in the Minkowski space-time.   
 In low dimensions, namely two, this symmetry happens to reduce to
 very strict requirements with a well-known mathematical structure
 described by the Virasoro algebra. Investigation of the 
 properties of such algebras leads to amazing results
 and progresses in the area. The Virasoro generators are 
 moreover the modes of the stress-energy tensor, which 
 generates space-time diffeomorphisms of the theory. 
 As a consequence, a two-dimensional conformal field
 theory is basically a quantum field theory endowed with
 a stress-energy tensor whose generators must satisfy 
 specific algebraic properties and commutation relations.
 Also, the theory contains a special class of fields,
 the ``primary fields'', whose transformations properties 
 are very much related to how these fields commute with the 
 stress-energy tensor itself.
 
 Interestingly enough, conformal symmetries can be found
 very often in actual physical systems. Most of the 
 times this goes along with scaling invariance and,
 although the two properties do not coincide, they 
 are nevertheless very often interchanged. Models 
 with no proper scale dimensions, as for example
 massless models, are usually conformally invariant 
 and form the prototypes we can look at, not to mention 
 the huge amount of results, models and features carried
 by string theory, which is the straightforward 
 application of conformal field theory. However, within 
 the already mentioned two-dimensional models, 
 a special class is given
 by the so called chiral theories, a group of models
 where the fields only depend on the ``light-cone'' variables
 $x^{\pm}\coloneqq x^0\pm x^1$.
 Those theories decouples into two copies of 
 singular theories, either of them being concerned with 
 the one variable $x^+$ or $x^-$, respectively. This 
 means that the whole business reduces to a one-dimensional 
 theory, and the original model can be reconstructed eventually
 taking the tensor product of the two one-dimensional copies. 
 The term chiral becomes then synonym of one-dimensional 
 world living on a light-ray:
 \begin{figure}[htbp]
 \centering 
 \begin{tikzpicture}[scale=1.5]
 \draw [->] (-1,0) --(1,0);
 \draw [->] (0,-1) --(0,1);
 \draw [dashed] (-0.75,-0.75) -- (0.75,0.75);
 \draw [dashed] (0.75,-0.75) -- (-0.75,0.75);
 \node at (1,-0.2) {\footnotesize $x^1$};
 \node at (-0.2,1) {\footnotesize $x^0$};
 \node[rotate around={45:(-0.6,2)}] 
 {\footnotesize $x^+$};
 \node[rotate around={-45:(0.5,1.8)}] 
 {\footnotesize $x^-$};
 \end{tikzpicture}
 \end{figure}
 
 \noindent Each real line supports both the time-like 
 property (positivity of the energy) and the
 space-like commutativity (causality). Moreover 
 the real line can be taken onto the unit circle (minus 
 a point) via the Cayley transformations and thus 
 we shall basically be concerned with fields living 
 on a circle, 
 where the conformal transformations acquire the 
 form of general diffeomorphisms.
 
 \bigskip 
 Going back to the mathematical questions, we have 
 already stated that a revolutionising result 
 was found by Tomita and Takesaki and undergoes 
 the name of modular theory. Starting with a von 
 Neumann algebra and a cyclic and separating state 
 one can automatically construct an inner group of 
 automorphisms $\sigma_t$ whose explicit form 
 depends on the algebra itself and on the state 
 provided. In some special case, where the algebras
 are generated by local fields localised in particular 
 space-time regions, this group of automorphisms 
 happens to coincide with some symmetry group 
 occurring in physics (Lorentz boosts, dilations). 
 This result opens a brand new horizon of questions, 
 because it seems that the space-time symmetries lie 
 behind the physical content, back in the algebraic 
 properties of the quantities at hand. It is tempting
 to generalise such results and further investigate them.
 The main content of this thesis is exactly modular theory 
 for Fermi fields in one dimension: in particular, we have 
 been looking at fields localised in disjoint intervals,
 trying to derive and explain the features of their modular 
 theory. It turns out that whenever we choose the fields 
 to be localised in many disjoint intervals, the action of 
 the modular group 
 introduces a mixing among those different intervals on top 
 of a geometric action moving the points, 
 (\cite{CH:2009}, \cite{LMR:2009}). 
 This result can be traced back to the existence of 
 a vacuum preserving isomorphism moving the fermions 
 all around the circle \cite{Rehren:2012wa}. We have
 widely exploited this feature considering different 
 representations of the algebras and different situations 
 at hand, varying the geometric positions of the intervals 
 and comparing the new results to previous statements.
 Besides modular theory itself, this work gave us a 
 deeper understanding of how Fermi fields behave on the
 circle.
 
 Also, since products of Fermi fields generate observables
 as currents and the stress-energy tensor, these subtheories 
 can be embedded via the mentioned isomorphism and new 
 characteristics emerge. Currents generate gauge transformations
 which are therefore delocalised all around the circle, as 
 well as new multi-local diffeomorphisms given by the 
 embedded stress-energy tensor. As a consequence, all the
 standard constructions we have for fermions and related 
 models can be rephrased in terms of this new aspect,
 giving rise to a new class of perspectives.
 
 \bigskip 
 As for the  
 organisation of the material, this thesis is divided into 
 different parts. In the beginning we provide the standard 
 description of the mathematical framework lying behind 
 algebraic quantum field theory, following the lines of 
 \cite{Haag}. We introduce the technical aspects of 
 von Neumann algebras and the world of conformal field 
 theory in the field theoretical setting. 
 
 We then move to the analysis of the modular theory for 
 fermions localised in different intervals, 
 showing the new
 aspects together with new insights on the standard 
 constructions. We ought to mention that part of the 
 ideas were triggered by the original work of 
 Casini and Huerta, \cite{CH:2009}, where the authors 
 calculated the modular group for fermions in disjoint 
 intervals using methods coming from density matrices 
 and statistical mechanics. We took their starting point 
 to rephrase everything in the language of algebraic quantum 
 field theory and operator algebras. Other ideas came from
 different works on boson-fermion correspondences, 
 \cite{Ang:2011} as well as others, and we
 tried to contribute attacking the problems from the 
 angle of local quantum physics.
 
 A third part describes the class of models which can 
 be obtained out of Fermi fields, mainly concerning
 currents and their features, in the light of the 
 new background provided. The multi-local features 
 restrict to these subalgebras with the help of suitable 
 gauge transformations, which can be related to the 
 diffeomorphisms covariance in a limpid way.
 
  
\endgroup			
