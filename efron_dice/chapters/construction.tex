
\section{Construction of fully perfect Efron's dice}\label{fullyPerfect}
The simplest case $Z=N$ provides a standard construction of
fully perfect Efron's dice, as we shall show by iteration. 
For $N=2$ the dice always tie, therefore we start with $N=3$.
As it will be later on clarified, a standard constuction may proceed
by cyclicity as follows: we pick a different starting die for each
face and fill the values up incrementing each face by $1$ upwards.
\begin{example}[Three three-sided perfect Efron's dice] 
The following choice 
\begin{align*}
D_1&=\set{\odot, \blank, \blank} \\
D_2&=\set{\blank, \blank, \odot} \\
D_3&=\set{\blank, \odot, \blank} 
\end{align*}
allows
\begin{align*}
D_1&=\set{1,6,8} \\
D_2&=\set{3,5,7} \\
D_3&=\set{2,4,9} 
\end{align*}
\end{example}
with winning probability $\p{I}{I+1}=5/9$. It is easy to see
that any other choice would lead to the same result.
\begin{example}[Four four-sided perfect Efron's dice] 
The following choice 
\begin{align*}
D_1&=\set{\odot, \blank, \blank, \blank} \\
D_2&=\set{\blank, \blank, \odot, \blank} \\
D_3&=\set{\blank, \odot, \blank, \blank} \\
D_4&=\set{\blank, \blank, \blank, \odot}
\end{align*}
allows
\begin{align*}
D_1&=\set{1, 7, 10, 16} \\
D_2&=\set{4, 6, 9, 15} \\
D_3&=\set{3, 5, 12, 14} \\
D_4&=\set{2, 8, 11, 13}
\end{align*}
with winning probability $\p{I}{I+1}=9/16$.
\end{example}
\begin{example}[Five five-sided perfect Efron's dice] 
Along the same lines
\begin{align*}
D_1&=\set{1, 7, 13, 18, 25} \\
D_2&=\set{5, 6, 12, 19, 24} \\
D_3&=\set{4, 10, 11, 20, 23} \\
D_4&=\set{3, 9, 15, 16, 22}\\
D_5&=\set{2, 8, 14, 17, 21}
\end{align*}
with winning probability $\p{I}{I+1}=14/25, I = 3,4$ and 
$\p{I}{I+1}=13/25$ elsewise.
\end{example}
\begin{example}[Six six-sided perfect Efron's dice] 
Also 
\begin{align*}
D_1&=\set{1, 9, 14, 24, 28, 35} \\
D_2&=\set{6, 8, 13, 23, 27, 34} \\
D_3&=\set{5, 7, 18, 22, 26, 33} \\
D_4&=\set{4, 12, 17, 21, 25, 32} \\
D_5&=\set{3, 11, 16, 20, 30, 31}\\
D_6&=\set{2, 10, 15, 19, 29, 36}
\end{align*}
with winning probability $\p{I}{I+1}=20/36=5/9$.
\end{example}
With no need to go any further, the probability path is 
easy to recognise. In fact, for each $N$ we have at most
$N(N+1)/2 -1$ mutual winning combinations, giving rise to 
\[
\left.\p{I}{I+1}\right|_{\textrm{max}}=\frac{N(N+1)/2 -1}{N^2}=\frac{1}{2}+\frac{1}{2N}-\frac{1}{N^2}
\]
which decreases monotonously to $1/2$ as $N\to\infty$. The maximal winning
probability is obtained for $N=4$.

\bigskip
The package \texttt{Rdice}\footnote{\url{https://cran.r-project.org/web/packages/Rdice/index.html}} provides the function \texttt{is.nonTransitive}
to check whether a given set of dice is non-transitive; as an 
example we can prove if on the above $N=5$ fully Efron's set 
as follows:

\begin{code}
fullyPerfectFive <- data.frame(
	  die1 = c(1,7,13,18,25),
	  die2 = c(5,6,12,19,24),
	  die3 = c(4,10,11,20,23),
	  die4 = c(3,9,15,16,22),
	  die5 = c(2,8,14,17,21)
	)
	
> is.nonTransitive(fullyPerfectFive)
[1] TRUE
\end{code}


\subsection{The case $Z< N$: minimal winning probabilities}
We will now look at the case $Z<N$; in particular we shall
try to generate non-transitive set of dice with \emph{minimal}
winning probability.

\bigskip
It is straightforward to see that once we fix the number of 
dice $Z$, the minimal winning probabilities one can achieve
are given by
\begin{equation}
\left. \tilde{p}(N) \equiv p(N)\right|_{\textrm{min}} = 
	\begin{cases}
		\dfrac{\dfrac{N^2}{2}+1}{N^2},\quad N\textrm{ even}\\[2em]
		\dfrac{\dfrac{N^2-1}{2}+1}{N^2},\quad N\textrm{ odd}.
	\end{cases}\label{minimalProb}
\end{equation}
In particular, the minimal probability of any given odd integer $N$
is always smaller than the one achieved by its next even integer, 
namely one has $\tilde{p}(2K-1)<\tilde{p}(2K),\, K\in\N$. As such,
for the minimal probability examples we will only be looking at 
the case of $N$ odd.

\bigskip 
The package \texttt{}{Rdice} provides the function \texttt{nonTransitive.generator},
that generates a random set of non-transitive dice with given 
probability, once one specifies conditions on the total number of
faces and maximum integer value allowed for any thereof. For the sake 
of simplicity we can start looking at the case of $Z=3$ and 
vary $N$ in order to minimise the winning probability according 
to \eqref{minimalProb}. For $N=5$ (we look at odd integers only, as specified before)
one has $\tilde{p} = 13/25$: with \texttt{nonTransitive.generator(dice = 3, 
faces = 5, max\_value = 6, prob = 13/25)}

\begin{code}
# set prob = 13/25
is.nonTransitive(
	data.frame(
	die1 = c(0,2,6,1,6),
	die2 = c(5,4,0,1,4),
	die3 = c(2,4,2,3,3)
	), 
prob = 13/25
)	
[1] TRUE
\end{code}

Choosing different values for $N$ generates non-transitive sets with
smaller winning probabilities, as $N$ increases, due to the fact that 
\eqref{minimalProb} decreseas monotonously for $N\to\infty$; with $N=13$ faces 
and using the function 
\texttt{nonTransitive.generator(dice = 3, faces = 13, max\_value = 20)}
we find the smallest possible winning probability for a non-transitive 
set of $Z=3$ dice to be given by:
\begin{code}
is.nonTransitive(
	data.frame(
	die1 = c(5,8,11,6,6,8,7,18,5,4,16,18,16),
	die2 = c(3,11,0,4,17,15,0,13,2,14,17,11,19),
	die3 = c(16,0,8,9,4,9,11,17,12,12,9,1,18)
	),
)	
[1] TRUE
\end{code}
where the second die beats the third with $\tilde{p}^*=84/169 = 0.5029586$.
The other combinations, however, beat each other with $p>\tilde{p}^*$.
For a set of $Z=4$ we have produced the following results: $N=7$-faced
dice 
\begin{code}
is.nonTransitive(
	data.frame(
	die1 = c(3,6,8,3,9,0,2),
	die2 = c(8,9,2,1,1,8,1),
	die3 = c(6,5,6,0,6,4,7),
	die4 = c(4,0,0,4,10,9,4)
	),
)	
[1] TRUE
\end{code}
with the first dice realising a non-transivitive winning 
probability of $25/49$. For $N=8$ we reach the least possible
non-transitive winning probability, namely, due to ties,
\begin{code}
is.nonTransitive(
	data.frame(
	die1 = c(5,5,7,10,4,4,5,1),
	die2 = c(10,2,2,5,6,2,6,3),
	die3 = c(0,1,9,1,0,10,10,9),
	die4 = c(6,7,0,7,0,7,5,6)
	),
)
[1] TRUE
\end{code}
presents the first and the third dice winning the 
corresponding next one with $p=1/2$.
