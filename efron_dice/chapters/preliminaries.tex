\section{Preliminaries}
Let $D_1,\ldots,D_{Z}$ be $Z$ finite sets, 
where $D_k=\set{d_k^1\leq\ldots \leq d_k^N}$, $k=1,\ldots,Z$.
Should the elements of such sets be natural numbers, we would then refer
to the collection of $D_1,\ldots,D_{Z}$ as to a set of dice, 
whose cardinalities $\abs{D_k}$ denote the number of faces thereof. 
As to do so, picking a random element $d_k^P$ will be the equivalent
of rolling the $k^{\textrm{th}}$ die on its $P^{\textrm{th}}$ 
face with outcome $d_k^P$. A game of two dice $(I,J)$ is the cartesian product $D_I\times D_J$. Likewise, a game of $Z$ dice is the pairwise union of all the
possible $\binom{Z}{2}$ games of two dice each, with cardinality $\binom{Z}{2}\cdot N^Z$.

\bigskip
Let us henceforth consider a set of $Z$ dice. Without loss of generality we may
assume $N$ to be even (though the odd case works along the same lines). 
For each pair of dice $(D_I,D_J)$ let 
$M_{IJ}\coloneqq \abs{\set{(i,j)\mid d_I^i>d_J^j}}$. The die $D_I$ is said
to win against the die $D_J$ (denoted as $D_J\vartriangleright D_J$)    
whenever $M_{IJ}>N/2$, otherwise the two dice are
said to tie. Replacing $I\to J$ exhausts the one more remaining case. In a nutshell,
the former die wins if it rolls higher than the latter more than a half times; in particular, counting all the possible outcomes, the probability of such to be achieved is given by the overall possible combinations 
\[
\p{I}{J}=\frac{M_{IJ}}{N^2}
\]
\begin{example}
Given two three-sided dice as 
\begin{align*}
D_1&=\set{1,5,6} \\
D_2&=\set{2,3,4} 
\end{align*}
the former dice wins against the latter $M_{12}=6$ times,
hence with $\p{1}{2}=2/3$.
\end{example}
\begin{definition}
A set of $Z$ $N$-sided dice is said to be fully Efron's of order $(Z,N)$ 
if the cyclicity condition
\[
D_1\vartriangleright D_2\vartriangleright \ldots \vartriangleright
D_Z\vartriangleright D_1
\]
holds with $\p{I-1}{I}=\p{I}{I+1}, \quad I=2,\ldots,Z-1$. Should the 
mutual winning probabilities be different from one other, such set is 
then said to be partially Efron's. Also, a fully Efron's set of dice
of order $(Z,N)$ is said to be perfect if the dice entries exhaust
all the natural numbers $\set{1,\ldots,ZN}$.
\end{definition}