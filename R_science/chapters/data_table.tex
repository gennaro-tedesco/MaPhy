%************************************************
\chapter{The \pkgname{data.table} package}\label{sec: datatable}
%************************************************
\texttt{install.packages(`data.table')}\\
\texttt{library(`data.table')}
\bigskip 

A \texttt{data.table} is a \df plus
additional features that allow to strongly simplify
a large set of operations on data as subsetting 
according to constraints, grouping by specific
categories, behaviours and functions as well as
easy joins between them based on different 
common keys and values. As such, we recommend as
best practice to always transform any set of data
into such format first and then start performing
anything, as they are the most memory efficient.
\bigskip 

A \texttt{data.table} does not have row numbers
because they are deprecated, as joins and subsets
must occur on keys and common values insted.
As such, one assigns:
\begin{verbatim}
iris <- data.table(iris)
R: key(iris)
NULL

R: setkey(iris, Species)
R: key(iris)
[1] "Species"

R: dim(iris)
[1] 150   5
\end{verbatim}
Multiple keys can be set and the data table 
will be sorted accordingly (also refer 
here\footnote{ 
\url{http://stackoverflow.com/a/20057411/5017267}
}). For instance:
\begin{verbatim}
R: setkey(iris, Species, Sepal.Length) 

     Sepal.Length Sepal.Width Petal.Length Petal.Width   Species
  1:          4.3         3.0          1.1         0.1    setosa
  2:          4.4         2.9          1.4         0.2    setosa
  3:          4.4         3.0          1.3         0.2    setosa
  4:          4.4         3.2          1.3         0.2    setosa
  5:          4.5         2.3          1.3         0.3    setosa
\end{verbatim}
If we want the opposite (descending) order
\begin{verbatim}
R: setorder(iris, Species, -Sepal.Length)

     Sepal.Length Sepal.Width Petal.Length Petal.Width   Species
  1:          5.8         4.0          1.2         0.2    setosa
  2:          5.7         4.4          1.5         0.4    setosa
  3:          5.7         3.8          1.7         0.3    setosa
  4:          5.5         4.2          1.4         0.2    setosa
  5:          5.5         3.5          1.3         0.2    setosa
\end{verbatim}
Also note:
\begin{verbatim}
R: names(iris)
[1] "Sepal.Length" "Sepal.Width"  "Petal.Length" 
	"Petal.Width"  "Species"

R: setnames(iris, c("Sepal.Length","Sepal.Width"), 
		  c("sep_length", "sep_width"))
R: names(iris)
[1] "sep_length"   "sep_width"    "Petal.Length" 
		"Petal.Width"  "Species" 
\end{verbatim}

\section{Subsetting a data table upon constraints}
The general syntax form for a data table is 
\texttt{dt[i,j, by = k]}, meaning to subset the rows using 
\texttt{i}, then apply \texttt{j} as a function grouped by
\texttt{k}. The syntax is totally equivalent to the 
standard \texttt{SQL}, with \texttt{i,j} replacing 
\texttt{where} and \texttt{select} clauses, respectively.
Any formal expression or function can be used 
as \texttt{j}.
\bigskip 

Columns in a data table can be accessed by name 
or by position reference: the two methods below are
indeed equivalent in the result (notice 
\texttt{with = FALSE} in the latter)
\begin{verbatim}
R: iris[, .(Species, Petal.Width, Petal.Length)]
R: iris[, c(5,4,3), with = FALSE]

       Species Petal.Width Petal.Length
  1:    setosa         0.2          1.4
  2:    setosa         0.2          1.4
  3:    setosa         0.2          1.3
  4:    setosa         0.2          1.5
  5:    setosa         0.2          1.4 
\end{verbatim}
\bigskip 

Equivalently, new columns can be defined and added
with
\begin{verbatim}
R: iris[, .(new_value = Sepal.Length/Sepal.Width, Species)]

     new_value   Species
  1:  1.457143    setosa
  2:  1.633333    setosa
  3:  1.468750    setosa
  4:  1.483871    setosa
  5:  1.388889    setosa
\end{verbatim}
and can be deleted as
\texttt{iris[, c(`Petal.Width',`Petal.Length') := NULL]}.
\bigskip

Rows can be subset according to constraints:
\begin{verbatim}
subset(iris, 
       Species == 'virginica'
       & (Petal.Width > 2.3 | Sepal.Width < 3)) 
       
subset(iris,
       Species %in% c(`virginica',`setosa'))
\end{verbatim}
The \texttt{not in} operator in \texttt{R} is obtained
by negating the variable instead of negating 
the set it belongs to, i. e. 
\texttt{subset(iris,
      !Species \%in\% c(`virginica',`setosa'))}
gives the subset of variable whose Species are neither 
\emph{virginica} nor \emph{setosa}. The above are 
equivalent to directly impose the subset 
on the rows as
\begin{verbatim}
R: iris[Species == `virginica']
R: iris[Species %in% c(`virginica',`setosa')]
R: iris[!Species %in% c(`virginica',`setosa')]
\end{verbatim}
\bigskip 

The \texttt{.I} operator shows the row number in 
correspondence of a matching constraint, should we 
want the data to be grouped by some other variables.
For example
\begin{verbatim}
R: iris[, .I[Petal.Length = max(Petal.Length)], 
		by = Species]

      Species  V1
1:     setosa   1
2: versicolor  55
3:  virginica 106
\end{verbatim}
The variable \texttt{V1} contains the actual row number 
at the match, which in turnI can be plugged in the data 
table again, by reference, to have back the entire rows 
in correspondece thereof 
\begin{verbatim}
R: iris[iris[, .I[Petal.Length = max(Petal.Length)], 
		by = Species]$V1]

   Sepal.Length Sepal.Width Petal.Length Petal.Width    Species
1:          5.1         3.5          1.4         0.2     setosa
2:          6.5         2.8          4.6         1.5 versicolor
3:          7.6         3.0          6.6         2.1  virginica
\end{verbatim}

Simple frequencies counts per group can be obtained 
via the \texttt{.N} operator.
\begin{verbatim}
R: iris[, .N ,by = Species]

      Species  N
1:     setosa 50
2: versicolor 50
3:  virginica 50
\end{verbatim}
The \texttt{.SD} operator creates a data table whose 
values are the original values except the variables 
grouped by. Such new data table can be accessed on the 
fly to perform operations upon:
\begin{verbatim}
R: iris[, lapply(.SD, sd), by = Species]

      Species Sepal.Length Sepal.Width Petal.Length Petal.Width
1:     setosa    0.3524897   0.3790644    0.1736640   0.1053856
2: versicolor    0.5161711   0.3137983    0.4699110   0.1977527
3:  virginica    0.6358796   0.3224966    0.5518947   0.2746501
\end{verbatim}
\bigskip 

The \texttt{[} operator can be used in sequence to allow 
partial groupings as in the example below:
\begin{verbatim}
# simple use would be:
groups <- quarks[,
                 .N,
                 keyby = c("flavour", "lab")
                 ]
                 
R: head(groups)

   flavour lab N
1:  bottom   B 3
2:  bottom   C 3
3:  bottom   D 2
4:  bottom   E 4
5:  bottom   F 6
6:  charme   A 1

# we can normalise each count N to
# the overall number of counts per 
# flavour, piping the [
groups <- quarks[,
                 .N,
                 keyby = c("flavour", "lab")
                 ][,
                   c("flavour_sum", "lab_freq") := 
                       list(sum(N), N/sum(N)),
                   by = "flavour"
                   ]

R: head(groups)

   flavour lab N flavour_sum   lab_freq
1:  bottom   B 3          18 0.16666667
2:  bottom   C 3          18 0.16666667
3:  bottom   D 2          18 0.11111111
4:  bottom   E 4          18 0.22222222
5:  bottom   F 6          18 0.33333333
6:  charme   A 1          17 0.05882353                  
\end{verbatim}



\section{Random and unique rows}
In order to show the power of random and distinct
sampling we will make use of the \texttt{quarks} data table as 
defined in (\ref{quark}). Random samples are easily obtained:
\begin{verbatim}
R: set.seed(1)
R: quarks[sample(.N,10)]

    lab flavour  S_z
 1:   A strange  1/2
 2:   E strange -1/2
 3:   B strange  1/2
 4:   B  bottom  1/2
 5:   E strange  1/2
 6:   B    down  1/2
 7:   C      up  1/2
 8:   B  bottom  1/2
 9:   D      up  1/2
10:   F    down -1/2
\end{verbatim}
The function \texttt{unique(dt, by = c(first, second))}
allows to fetch the \emph{first occurrence} of unique
row according to the variables ``first, second''.
\begin{verbatim}
R: unique(quarks, by = c("flavour"))

   lab flavour S_z
1:   A strange 1/2
2:   B  bottom 1/2
3:   B    down 1/2
4:   C      up 1/2

R: unique(quarks, by = c("flavour", "S_z"))

   lab flavour  S_z
1:   A strange  1/2
2:   E strange -1/2
3:   B  bottom  1/2
4:   B    down  1/2
5:   C      up  1/2
6:   F    down -1/2
\end{verbatim}

\section{Grouping by}
Variables can by grouped by according to
the following grammar:
\begin{verbatim}
R: iris[,.(width = mean(Petal.Width), dev = sd(Petal.Width), .N), 
        by = Species]
        
      Species  mean       dev  N
1:     setosa 0.246 0.1053856 50
2: versicolor 1.326 0.1977527 50
3:  virginica 2.026 0.2746501 50


R: quarks[,.(observations = .N), 
	keyby = c("lab", "flavour")]

    lab flavour observations
 1:   A  charme            1
 2:   A    down            3
 3:   A strange            5
 4:   A     top            2
 5:   B  bottom            3
 6:   B  charme            3
\end{verbatim}

\section{Joining data tables}
\subsubsection*{Inner joins}
Given two data tables having at least one common
variable, the syntax
\texttt{merge(first, second, by = c(`var1', `var2'))}
performs inner join based on the variables ``var1, var2''.
In the following example different laboratories
perform measures of the spin projections of 
different quarks. We want to find what each lab
has measured when the same quark has appeared:
\begin{verbatim}
set.seed(10)
first <- quarks[sample(.N, 10)]

set.seed(20)
second <- quarks[sample(.N, 10)]

       first        |      second
--------------------------------------
   lab flavour  S_z | lab flavour  S_z
1:   D  bottom  1/2 |   C    down  1/2
2:   E  charme -1/2 |   F strange  1/2
3:   C strange -1/2 |   F strange -1/2
4:   C strange  1/2 |   A     top  1/2
5:   D strange -1/2 |   D      up -1/2
6:   ............   |  ..............

R: merge(first, second, by = c("lab", "flavour"))

   lab flavour S_z.x S_z.y
1:   B strange   1/2  -1/2
2:   C strange  -1/2   1/2
3:   C strange   1/2   1/2
4:   C strange   1/2   1/2
5:   D  bottom   1/2   1/2
6:   F  bottom   1/2   1/2
\end{verbatim}
The above can equivalently obtained with the  
syntax \texttt{first[second, nomatch=0]} once we set the 
variables we want to join on as keys. In fact
\begin{verbatim}
R: setkey(first, lab, flavour)
R: setkey(second, lab, flavour)
R: first[second, nomatch = 0]

   lab flavour  S_z i.S_z
1:   B strange  1/2  -1/2
2:   C strange -1/2   1/2
3:   C strange  1/2   1/2
4:   C strange  1/2   1/2
5:   D  bottom  1/2   1/2
6:   F  bottom  1/2   1/2
\end{verbatim}
gives the same results, as also \texttt{second[first, nomatch=0]}.
The condition \texttt{nomatch=0} ensure the inner join as 
all the non-matching rows get discarded.

\subsubsection*{Left joins}
The two equivalent give the same results:
\begin{verbatim}
# notice all.x = TRUE
R: merge(first, second, by = c("lab", "flavour"), 
         all.x = TRUE)[order(flavour)]
                        
# notice nomatch = 0 has been taken off
R: second[first]   

    lab flavour  S_z i.S_z
 1:   D  bottom  1/2   1/2
 2:   F  bottom  1/2   1/2
 3:   B  charme   NA   1/2
 4:   C  charme   NA  -1/2
 5:   E  charme   NA  -1/2
 6:   B strange -1/2   1/2
 7:   C strange  1/2  -1/2
 8:   C strange  1/2   1/2
 9:   C strange  1/2   1/2
10:   D strange   NA  -1/2
\end{verbatim}
The advantage of the latter is that ordering is 
automatically performed on keys.

\subsubsection*{Full joins}
Full joins are given by
\texttt{merge(first, second, by = c("flavour", "S\_z"), 
all = TRUE)}

\subsubsection*{Cartesian products}
In order to perform cartesian products 
we refer to the non-in-built function 
shown in (\ref{sec: functions}) 
\texttt{cross.join(first, second)}.


