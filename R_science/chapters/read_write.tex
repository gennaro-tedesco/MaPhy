%************************************************
\chapter{Writing and reading data}\label{sec: read_write}
%************************************************
Below is an example on how to write out and read 
in data sets. 
\begin{verbatim}
set.seed(10)
mtcars <- data.table(mtcars)
cars   <- mtcars[sample(.N,5), sample(11,4), with = FALSE]

write.table(cars, file = "my_file.csv", sep = "\t", 
            quote = FALSE, append = FALSE, na = "NA", 
            dec = ".", row.names = FALSE)

read_cars <- read.table("my_file.csv", sep = "\t", quote = "",
                        header = TRUE, dec = ".", fill = FALSE, 
                        na.strings = c("NA", "-"), 
                        stringsAsFactors = FALSE)
                        
R: cars                  | R: read_cars

    disp carb gear drat  |    disp carb gear drat
1: 440.0    4    3 3.23  |1: 440.0    4    3 3.23
2: 167.6    4    4 3.92  |2: 167.6    4    4 3.92
3: 275.8    3    3 3.07  |3: 275.8    3    3 3.07
4: 120.1    1    3 3.70  |4: 120.1    1    3 3.70
5: 108.0    1    4 3.85  |5: 108.0    1    4 3.85        
\end{verbatim}
The option \texttt{na.strings = c("NA", "-")} decides
which lines must be interpreted as \texttt{NA}. 
Likewise \texttt{fill = TRUE} allows to skip and 
continue whenever inconsistencies in the data are
present: on the other hand \texttt{fill = FALSE} 
throws an error whenever so (and hence allows 
control on the inconsistent data). To trim leading 
and tailing space from unquoted strings use 
\texttt{strip.white = TRUE}.
\bigskip

The \texttt{data.table} package makes use of 
\texttt{fread} to read data file in, this being
much faster (especially for large sets of data),
while keeping the same syntax.
