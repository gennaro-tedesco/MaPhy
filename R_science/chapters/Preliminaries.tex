%************************************************
\chapter{Introduction}\label{sec: pre}
%************************************************
\noindent In the following a quick guide to
common features of some \texttt{R} packages
is shown. It is by no means intended to be a guide
to the language, rather an introductory
primer to some libraries useful in data 
science for the ones who are already 
familiar with the language. 

As best practice, the reader is always
addressed to the official documentation; 
moreover, given any \texttt{R} function, the 
line \texttt{?<function>} prompts the corresponding
definitions and in-built help.
\bigskip

A full list of functions according to the package 
they are defined in is available here.\footnote{ 
\url{http://www.rdocumentation.org/}}

\bigskip 

Unless specified otherwise, minimal working examples
are shown by making use of the sample datasets provided
by \texttt{R}, in particular we will mainly refer to
\texttt{data(iris)}, \texttt{data(mtcars)} and \texttt{data(morley)}. 
We will henceforth refer to a generic data frame as to \texttt{df}.
Another set of data we will make use of is the following
quark\label{quark} data set: 
\begin{verbatim}
set.seed(1)
lab      <- sample (LETTERS[1:6], 100, replace = TRUE) 
flavour  <- sample(c("up", "down", "charme", "strange", 
		"top", "bottom"), 100, replace = TRUE)
S_z      <- sample(c("1/2", "-1/2"), 100, replace = TRUE)
quarks   <- data.table(lab, flavour, S_z)

R: head(quarks, 5)

   lab flavour  S_z
1:   B strange  1/2
2:   C  charme  1/2
3:   D    down -1/2
4:   F  bottom  1/2
5:   B strange  1/2
\end{verbatim}

