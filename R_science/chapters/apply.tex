%************************************************
\chapter{Vectorised operations on data frames}\label{sec: apply}
%************************************************
Given a data frame, the functions of the family
\texttt{apply} allow to perform vectorised operations
on rows and columns thereof, without having the manually
access each entry to loop through.

\section{apply}
The general wrapper for such operations is the
\texttt{apply} function, where rows and columns can 
be accessed specifying the labels $(1,2,\ldots)$ (and higher 
if any other multi-dimensional object is contained 
therein). It returns a vector (or a list) of 
values obtained by applying a function to the 
rows (columns, respectively) of a data frame, coerced
to matrix first. The general syntax is 
\texttt{apply(df, margin, FUN = <function> )}, with \texttt{margin} being
$1,2,\ldots$ and \texttt{<function>} being any function.
\begin{verbatim}
R: cars <- head(mtcars[,1:7],4)
R: cars
                   mpg cyl disp  hp drat    wt  qsec
Mazda RX4         21.0   6  160 110 3.90 2.620 16.46
Mazda RX4 Wag     21.0   6  160 110 3.90 2.875 17.02
Datsun 710        22.8   4  108  93 3.85 2.320 18.61
Hornet 4 Drive    21.4   6  258 110 3.08 3.215 19.44

# choose 1 for rows
R: apply(cars, 1, mean)

Mazda RX4  Mazda RX4 Wag     Datsun 710 Hornet 4 Drive 
45.71143       45.82786       36.08286       60.16214 
      
# choose 2 for columns
R: apply(cars, 2, mean)  

    mpg      cyl     disp       hp     drat       wt     qsec 
21.5500   5.5000 171.5000 105.7500   3.6825   2.7575  17.8825
\end{verbatim}
In the simple cases of the function being the sum or the mean,
specific operators exist as \texttt{colSums} and \texttt{colMeans},
and equivalently for rows:
\begin{verbatim}
R: colSums(cars)

   mpg    cyl   disp     hp   drat     wt   qsec 
 86.20  22.00 686.00 423.00  14.73  11.03  71.53
 
R: rowMeans(cars)

Mazda RX4  Mazda RX4 Wag     Datsun 710 Hornet 4 Drive 
45.71143       45.82786       36.08286       60.16214 
\end{verbatim}

\section{lapply and sapply}
\texttt{lapply} applies a function to each element of a list
and returns a list back. Equivalently, \texttt{sapply} does
the same job but returns a vector back instead. As a data 
frame can be seen as a list of columns, one can have
\begin{verbatim}
R: head(quarks)

   lab flavour  S_z
1:   B strange  1/2
2:   C  charme  1/2
3:   D    down -1/2
4:   F  bottom  1/2
5:   B strange  1/2
6:   F    down -1/2

R: lapply(quarks, mode)
$lab
[1] "C"

$flavour
[1] "strange"

$S_z
[1] "1/2"

R: sapply(quarks, mode)

lab   flavour       S_z 
"C" "strange"     "1/2"
\end{verbatim}

\section{Vectorised assignments}
Vectorised assignments in \texttt{R} commute with 
functions, namely the operator \texttt{c} is such
that \texttt{c(f) = f(c)}.
\begin{verbatim}
f <- function(x) sin(x) - cos(2*x)

set.seed(1234)
x <- f(c(rnorm(5)))

set.seed(1234)
y <- c(f(rnorm(5)))

x == y

[1] TRUE TRUE TRUE TRUE TRUE
\end{verbatim}
Likewise, \texttt{ifelse} evaluates a given 
condition on each element of a vector, thus 
replacing an entire loop: the two examples 
below are indeed equivalent, the latter sparing
memory and being faster
\begin{verbatim}
set.seed(1234)
x <- rnorm(5)

R: for(i in seq(1:length(x))){
      if(x[i] < 0){
        print("negative")
      } else {
        print("positive")
      }
    }

[1] "negative"
[1] "positive"
[1] "positive"
[1] "negative"
[1] "positive"


R: ifelse(x < 0, "negative", "positive")

[1] "negative" "positive" "positive" "negative" "positive"
\end{verbatim}
The function \texttt{cumsum} returns the cumulative 
sums of values elementwise, easily replacing a 
loop through:
\begin{verbatim}
set.seed(1234)
x <- rnorm(5)

R: x
[1] -1.2070657  0.2774292  1.0844412 -2.3456977  0.4291247

R: cumsum(x)
[1] -1.2070657 -0.9296365  0.1548047 -2.1908930 -1.7617683
\end{verbatim}
\enlargethispage{\baselineskip}
As a more general result, given a vector of 
values, the operator \texttt{Reduce}
applies a function on pairs of values at 
a time, defined as
\begin{verbatim}
Reduce(f, x) = ...f(f(f(x[1],x[2]),x[3]),...)

set.seed(1234)
x <- rnorm(5)
R: x
[1] -1.2070657  0.2774292  1.0844412 -2.3456977  0.4291247
R: Reduce(max, x, accumulate = TRUE)
[1] -1.2070657  0.2774292  1.0844412  1.0844412  1.0844412
\end{verbatim}




